\documentclass[11pt]{article}
\usepackage[sc]{mathpazo}
\usepackage{amsmath,pifont}
\usepackage{fullpage}
\usepackage[authoryear,sectionbib,sort]{natbib}
\linespread{1.7}
\usepackage[utf8]{inputenc}
\usepackage{lineno}

\usepackage{graphicx} 
\usepackage{tabularx,setspace}

\title{Density-dependent selection in evolutionary genetics: a lottery model of Grime's triangle}
\author{Jason Bertram $^{1,\ast}$ \\ 
Joanna Masel $^{1}$}

\date{}

\begin{document}

\maketitle

\noindent{}1. Department of Ecology and Evolutionary Biology, University of Arizona, Tucson, AZ 85721.

\noindent{}$\ast$ Corresponding author; e-mail: jbertram@email.arizona.edu.


\bigskip

\textit{Manuscript elements}: 

\bigskip

\textit{Keywords}: r/K selection, absolute fitness, eco-evo, interference competition.

\bigskip

\textit{Manuscript type}: Article. 
% Or e-article, note, e-note, natural history miscellany,
% e-natural history miscellany, comment, reply, symposium, or
% countdown to 150.

\bigskip

\noindent{\footnotesize Prepared using the suggested \LaTeX{} 
template for \textit{Am.\ Nat.}}

\linenumbers{}
\modulolinenumbers[3]

\newpage{}

\section*{Abstract}

Fitness is typically represented in heavily simplified terms in evolutionary genetics, often using constant selection coefficients. This excludes fundamental ecological factors such as dynamic population size or density-dependence from our most genetically-realistic treatments of evolution, a problem that inspired MacArthur's influential but problematic $r$/$K$ theory. Following in the spirit of $r$/$K$-selection as a general-purpose theory of density-dependent selection, but grounding ourselves empirically in ``primary strategy'' trait classification schemes like Grime's triangle, we develop a new model of density-dependent selection which revolves around territorial contests. To do so, we generalize the classic lottery model of territorial acquisition, which has primarily been used for studying species co-existence questions, to accommodate arbitrary densities. We use this density-dependent lottery model to predict the direction of trait evolution under different environmental conditions and thereby provide a mathematical underpinning for Grime's verbal scheme. We revisit previous concepts of density-dependent selection, including $r$ and $K$ selection, and argue that our model distinguishes between different aspects of fitness in a more natural and intuitive manner.

\newpage{}

``...the concept of fitness is probably too complex to allow of a useful mathematical development. Since it enters fundamentally into many population genetics considerations, it is remarkable how little attention has been paid to it.'' --- Warren J. Ewens, Mathematical Population Genetics I, 2004 

\section*{Introduction}

Evolutionary models differ greatly in their treatment of fitness. In models of genetic evolution, genotypes are typically assigned constant (or frequency-dependent) selection coefficients describing the change in their relative frequencies over time due to differences in viability. This considerably simplifies the mathematics of selection, facilitating greater genetic realism, and can be justified over sufficiently short time intervals \citep[p. 276]{ewens_2012}. However, the resulting picture of evolution does not include even basic elements of the ecological underpinnings of selection, including dynamic population size and density-dependence.

By contrast, models of phenotypic trait evolution represent the change in phenotypic abundances over time using absolute fitness functions which describe how those traits affect survival and reproduction in particular ecological scenarios. This approach is powerful enough to model eco-evolutionary feedbacks between co-evolving traits, but is generally problem-specific and restricted to only a few traits at a time.

Far less work has been done to generalize beyond particular traits or ecological scenarios to models of fitness that still capture key distinctions between different forms of selection. Perhaps this is not surprising given that fitness is such a complex quantity, dependent on all of a phenotype's functional traits \citep{violle_2007} as well as its biotic and abiotic environment. In most cases, a detailed, trait-based, predictive model of fitness would be enormously complicated and have narrow applicability. It is therefore easy to doubt the feasibility of a simplified, general mathematical treatment of fitness \citep[p. 276]{ewens_2012}. For example, MacArthur's famous $r$/$K$ scheme \citep{macarthur_1962,macarthur_1967} is now almost exclusively known as a framework for understanding life-history traits, and judged on its failure in that role \citep{pianka_1970,stearns_1977,boyce_1984,reznick_2002}. The r/K scheme's original purpose was as an extension of the existing population-genetic treatment of selection to account for population density \citep{macarthur_1962}, but few attempts have been made to develop it further as a mathematical analysis of the major different forms of selection. 

Nevertheless, there are strong indications there are broader principles governing the operation of selection. In many groups of organisms, including corals \citep{darling_2012}, insects \citep{southwood_1977}, fishes \citep{winemiller_1992}, zooplankton \citep{allan_76} and plants \citep{grime_1988,westoby_1998}, different species can be divided into a small number of distinct trait clusters corresponding to fundamentally distinct ``primary strategies'' \citep{winemiller_2015}. The most famous example is Grime's plant trait classification scheme \citep{grime_1974,grime_1977,grime_1988}. Grime considered two broad determinants of population density: stress (persistent hardship e.g. due to resource scarcity, unfavorable temperatures or toxins) and disturbance (intermittent destruction of vegetation e.g. due to trampling, herbivory, pathogens, extreme weather or fire).  The extremes of these two factors define three primary strategies denoted by C/S/R respectively (Fig. \ref{fig:grimeschematic}): competitors ``C'' excel in low stress, low disturbance environments; stress tolerators ``S'' excel in high stress, low disturbance environments; and ruderals ``R''  excel in low stress, high disturbance environments. Survival is not possible in high-stress, high-disturbance environments. Grime showed that measures of C, S and R across a wide range of plant species are anti-correlated, so that strong C-strategists are weak S and R strategists, and so on. Thus, plant species can be classified on a triangular C/S/R ternary plot \citep{grime_1974}. Trait classification schemes for other organisms are broadly analogous to Grime's scheme \citep{winemiller_2015}. 

\begin{figure}
\centering
\includegraphics[scale=1]{grimeschematic.pdf}
\caption{\label{fig:grimeschematic} Schematic of Grime's triangle. The two axes show increasing levels of environmental stress and disturbance, respectively. Survival is not possible if the combination of stress and disturbance is too large (dashed line). This creates a triangle, each corner of which corresponds to a ``primary strategy''.} 
\end{figure}

Trait classification schemes show empirically that, beneath the complicated details of trait variation, even among closely-related species, fitness is predominantly determined by a few key factors such as intrinsic reproducive rate or stress-tolerance. However, while trait classification schemes are firmly grounded in trait data, they are verbal and descriptive rather than mathematical, a recognized hinderance to their broader applicability (e.g. \citealt{tilman_2007}). 

The aim of this paper is explore the interplay between some major dimensions of fitness in a simplified, spatially-homogeneous model of genotype growth, dispersal and competition. Building on the earlier r/K and C/S/R schemes, a central question is how fitness depends on the interaction between population density, intrinsic birth/death rates and competitive ability. 

We broadly follow the spirit of MacArthur's $r$/$K$ selection scheme in that our model is intended to account for fundamentally different forms of selection without getting  entangled in the intricacies of particular ecological scenarios. However, rather than building directly on MacArthur's formalism and its later extensions using Lotka-Volterra equations to incorporate competition (``$\alpha$-selection'') \citep{gill_1974,case_1974,joshi_2001}, our model is devised more with Grime's C/S/R scheme in mind, and represents a quantitative formalization of how C/S/R manifests at the level of within-population genotype evolution (as opposed to phenotypic divergence between species). This choice is motivated in part by the substantial empirical support for C/S/R-like schemes, and in part by the failings of the $r$/$K$ low/high density dichotomy --- many growth ability traits will confer advantages at both low and high densities (more details in the Discussion). 

As we will see, a generalized version of the classic lottery model of \cite{chesson_1981} is well suited for our purposes. In the classic lottery model, mature individuals (henceforth ``adults'') each require their own territory, whereas newborn individuals (henceforth ``propagules'') disperse to, and subsequently compete for, the territories made available by the death of adults. Territorial contest among propagules leaves a single victorious adult per territory, with the victor chosen at random from the propagules present (akin to a lottery), with probabilities weighted by a coefficient for each type representing competitive ability. This representation of competition is much simpler than having coefficients for the pairwise effects of types on each other (e.g. the $\alpha$ coefficients in the generalized Lotka-Volterra equations), or than modeling resource consumption explicitly \citep{tilman_1982}. The classic lottery model also has similarities with the Wright-Fisher model in population-genetics (section BLAH). 

However, the classic lottery model breaks down at low densitites, specifically if there are only  a few propagules dispersing to each territory (section ``Mean field approximation''). This was not a limitation in its original application to reef fishes, where a huge number of larvae from each species compete to secure territories each generation \cite{chesson_1981}, but is a critical limitation here. We analytically extend the classic lottery model to correctly account for low density behavior. 

%(each generation is a sample of constant size, with replacement, from the previous one)

In the section ``Model'', we introduce the basic assumptions of our generalized lottery model. Analytical expressions for the change in genotype abundances over time are introduced in section ``Mean field approximation'', with mathematical details relegated to the Appendices. The following two sections discuss the behavior of rare mutants and our treatment of Grime's triangle. 
 
\section*{Model}\label{sec:model}

We assume that each individual in a population requires its own territory to survive and reproduce (a site-occupancy model). All territories are identical, and the total number of territories is $T$. Time $t$ advances in discrete iterations, each representing the average time from birth to reproductive maturity. In iteration $t$, the number of reproductively mature individuals (henceforth called ``adults'') of the $i$'th genotype is $n_i(t)$, the total number of adults is $N(t)=\sum_i n_i(t)$, and the number of unoccupied territories is $U(t)=T-N(t)$. 

\begin{figure}
\centering
\includegraphics[scale=0.8]{lottery.pdf}
\caption{\label{fig:lottery} Each iteration of our lottery model has three main elements. First, propagules are produced by adults which are dispersed at random over the unoccupied territories (only propagules landing on unoccupied territories are shown). Lottery competition then occurs in each unoccupied territory (only competition in one territory is illustrated): each genotype has a probability proportional to $b_i n_i c_i$ of securing the territory. Then occupied territories are freed up by adult mortality. In Eq. \eqref{eq:delttot} and most of the paper, only adults can die (red crosses), but we will also consider the case where juveniles die (blue cross; section ``Primary strategies and Grime's triangle'').}
\end{figure}

Each iteration, adults produce $m_i$ new offspring (henceforth called ``propagules''). We assume adults cannot be ousted from occupied territories, and only propagules landing on unoccupied territories are included in $m_i$. Propagules disperse at random, independently of each other, and without spatial restrictions; each propagule has an equal probability of landing on any of the $U$ unnocupied territories. Thus, there is no interaction between propagules (e.g. avoidance of territories crowded with propagules). Loss of propagules during dispersal is subsumed into $m_i$ ($m_i$ only counts propagules which go on to contest the territory they land in). In general, $m_i$ will increase with $n_i$, and will also depend on population density $N$. For example, if $b_i$ is the number of propagules produced per individual from genotype $i$, then $m_i=b_i(1-N/T)n_i$ would reflect the loss of propagules due to dispersal to occupied territories. We assume $m_i=b_i n_i$, meaning that all propagules land on unoccupied territories (a form of directed dispersal). This choice is not biologically motivated; it simplifies the mathematics without seriously restricting the generality of our analysis, since the results presented here are not sensitive to the specific functional form of $m_i$. 

The number of individuals of the $i$'th genotype landing in any particular territory is denoted $x_i$. Random dispersal implies that in the limit $T\rightarrow \infty$, with $n_i/T$ held fixed, $x_i$ is Poisson distributed with mean territorial propagule density $l_i=m_i/U$ (this dispersal Poisson distribution is denoted $p_i(x_i)=l_i^{x_i} e^{-l_i}/x_i!$). Although $T$ is finite in our model, we assume that $T$ and the $n_i$ are large enough that $x_i$ is Poisson-distributed to a good approximation (details in Appendix A).  Note that the large $n_i$, large $T$ approximation places no restrictions on our densities $n_i/T$, but it does preclude consideration of demographic stochasticity when $n_i$ itself is very small (this will be discussed further in Section ``Invasion of rare genotypes and coexistence'').

When multiple propagules land on the same territory, they compete to secure the territory as they develop. This territorial contest is modeled as a weighted lottery: the probability that genotype $i$ wins a territory by the next iteration, assuming that at least one of its propagules is present, is $c_i x_i/\sum_j c_j x_j$, where $c_i$ is a constant representing relative competitive ability. 

The increase in $n_i$ over one iteration due to territorial acquisition, $\Delta_+ n_i$, is the sum of genotype $i$'s victories over all $U$ unoccupied territories. The expected proportion of unoccupied territories with $x_1,\ldots,x_G$ of the respective propagules is $p_1(x_1)\ldots p_G(x_G)$ ($G$ is the number of genotypes present), and the probability of $i$ winning in each of these territories is $c_i x_i/\sum_j c_j x_j$. As above, we assume that $T$ and the $n_i$ are large enough that we can ignore fluctuations in the proportion of unoccupied territories as well as the number of victories, which implies
\begin{equation}
\Delta_+ n_i(t)=U(t)\sum_{x_1,\ldots,x_G} \frac{c_i x_i}{\sum_j c_j x_j} p_1(x_1)\ldots p_G(x_G), \label{eq:growthsumuncoupled}
\end{equation}
where the sum only includes territories with at least one propagule present. 

In addition to propagule birth and competition, occupied territories become unoccupied due to mortality. For the majority of this manuscript we assume that mortality only occurs in adults (setting aside the deaths implicit in territorial contest), and at a constant, genotype-specific per-capita rate $d_i$, so that the overall change in genotype abundances is
\begin{equation}
\Delta n_i(t)=\Delta_+ n_i(t)-d_i n_i(t). \label{eq:delttot}
\end{equation}
This is reasonable approximation in the absence of disturbances; when we come to consider the effects of disturbances (Section ``Primary strategies and Grime's triangle''), we will incorporate disturbance-induced mortality in competing juveniles (Fig.~\ref{fig:lottery}). 

Note that the competitive ability coefficients $c_i$ represent a strictly relative aspect of fitness in the sense that they only influence population size $N$ indirectly by changing genotype frequencies; that may in turn change the population mean birth and death rates. This can be seen by summing Eq. \eqref{eq:delttot} over genotypes to get the  change in population size $N$, 
\begin{equation}
\Delta N=U(1-e^{-L})-\sum_i d_i n_i,\label{eq:deltN}
\end{equation}
which is independent of $c_i$ (here $L=\sum_j l_j$ is the overall propagule density).

\section*{Results}

\subsection*{Mean Field Approximation}

Eq. \eqref{eq:delttot} gives little intuition about the dynamics of density-dependent lottery competition, since \eqref{eq:growthsumuncoupled} involves an expectation over the random dispersal distributions $p_i$, which depend on how the $n_i$ change over time. We now evaluate this expectation using a ``mean field'' approximation; the intuition behind this approximation is as follows.

If the unoccupied territories are saturated with propagules from every genotype ($l_i\gg 1$ for all genotypes), the fluctuations in the $x_i$ are small compared to their means $l_i$ (since the $x_i$ are Poisson distributed), and so the composition of propagules in a territory will only rarely differ appreciably from the mean composition $l_1,l_2,\ldots,l_G$. Consequently, we can replace $x_i$ with $l_i$ in Eq. \eqref{eq:growthsumuncoupled}. This gives the classic lottery model \citep{chesson_1981},
\begin{equation}
\Delta_+ n_i(t)=U(t)\frac{c_i m_i}{\sum_j c_j m_j}= b_i n_i\frac{1}{L}\frac{c_i}{\overline{c}}, \label{eq:lottery}
\end{equation}
where $\overline{c}=\sum_j c_j m_j/M$ is the mean propagule competitive ability for a randomly selected propagule ($M=\sum_j m_j$ is the total number of propagules). 

However, in general the $l_i$ are not all large, and the $x_i$ cannot simply be replaced by their means in Eq. \eqref{eq:growthsumuncoupled}. Indeed, Eq. \eqref{eq:lottery} is nonsensical if $l_i$ is sufficiently small: genotype $i$ can win at most $m_i$ territories, yet Eq. \eqref{eq:lottery} demands a fraction $c_i m_i/\sum_j c_j m_j$ of the unoccupied territories $U$, no matter how large $U$ is. The source of this pathological behavior when $l_i\ll 1$ is that $x_i=1$ in the few territories where $i$ propagules do land, and so $i$'s growth comes entirely from territories which deviate appreciably from the mean.  

Our mean field approximation is similar to the high-$l_i$ approximation leading to Eq. \eqref{eq:lottery} in that we replace the $x_i$ with appropriate mean values. The key distinction is that territories with a single propagule from the focal genotype, whose behavior is critical at low densities, are handled separately. In place of the requirement of $l_i\gg 1$ for all $i$, our approximation only requires that there are no large discrepancies in competitive ability (specifically, that we do not have $c_i/c_j\gg 1$ for any two genotypes; further discussion in section ``Discussion''). We obtain (details in Appendix B)
\begin{equation}
\Delta_+ n_i(t)\approx b_i n_i\left[e^{-L}+(R_i+A_i)\frac{c_i}{\overline{c}}\right], \label{eq:master}
\end{equation}
where
\begin{equation}
R_i=\frac{\overline{c}e^{-l_i}(1-e^{-(L-l_i)})}{c_i +\frac{L-1+e^{-L}}{1-(1+L)e^{-L}}\frac{\overline{c}L- c_il_i}{L-l_i}},\label{eq:Dr}
\end{equation}
and
\begin{equation}
A_i=\frac{\overline{c}(1-e^{-l_i})}{\frac{1-e^{-l_i}}{1-(1+l_i)e^{-l_i}}c_il_i+\frac{1}{L-l_i}\left(L\frac{1-e^{-L}}{1-(1+L)e^{-L}}-l_i\frac{1-e^{-l_i}}{1-(1+l_i)e^{-l_i}}\right)\sum_{j\neq i}c_jl_j}.\label{eq:Da}
\end{equation}

Comparing Eq. \eqref{eq:master} to Eq. \eqref{eq:lottery}, the classic lottery per-propagule success rate $c_i/\overline{c}L$ has been replaced by three separate terms. The first, $e^{-L}$, accounts for propagules which land alone on unoccupied territories; these territories are won without contest. The second, $R_i c_i/\overline{c}$ represents competitive victories when the $i$ genotype is a rare invader in a high density population: from Eq. \eqref{eq:Dr}, $R_i\rightarrow 0$ when the $i$ genotype is abundant ($l_i\gg 1$), or other genotypes are collectively rare ($L-l_i\ll 1$). The third term, $A_ic_i/\overline{c}$, represents competitive victories when the $i$ genotype is abundant: $A_i\rightarrow 0$ if $l_i\ll 1$. The relative importance of these three terms varies with both the overall propagule density $L$ and the relative propagule frequencies $l_i/L$. If $l_i\gg 1$ for all genotypes, we recover the classic lottery model (only the $A_ic_i/\overline{c}$ term remains, and $A_i\rightarrow 1/L$). Thus, Eq. \eqref{eq:master} generalizes the classic lottery model to account for arbitrary propagule densities for each genotype. 

Fig.~\ref{fig:simcomp} shows that Eq. \eqref{eq:master} (and its components) closely approximate direct simulations of random dispersal and lottery competition over a wide range of propagule densities (obtained by varying $U$). Two genotypes are present, one of which has a $c$-advantage and is at low frequency. The growth of the low-frequency genotype relies crucially on the low-density competition term $R_i c_i/\overline{c}$, and also to a lesser extent on the high density competition term $A_i c_i/\overline{c}$ if $l_1$ is large enough (Fig.~\ref{fig:simcomp}b). On the other hand, $R_i c_i/\overline{c}$ is negligible for the high-frequency genotype, which depends instead on high density territorial victories (Fig.~\ref{fig:simcomp}d). 

\begin{figure}
\centering
\includegraphics[scale=0.7]{simulationcomparison.pdf}
\caption{\label{fig:simcomp} The change in genotype abundances in a density dependent lottery model is closely approximated by Eq. \eqref{eq:master}. $\Delta_+ n_i/m_i$ from Eq. \eqref{eq:master} (and its separate components) are shown, along with direct simulations of random dispersal and lottery competition over one iteration over a range of propagule densities (varied by changing $U$ with the $m_i$ fixed). Two genotypes are present. (a) and (b) show the low-frequency genotype with $c$-advantage ($m_1/M=0.1$, $c_1=1.5$), (c) and (d) show the high-frequency predominant genotype ($m_2/M=0.9$, $c_2=1$). Simulation points are almost invisible in (c) and (d) due to near exact agreement with Eq. \eqref{eq:master}.} 
\end{figure}

%with the exception of large discrepancies in competitive ability since it is extremely unlikely that a large discrepancy competitive would arise in the absence of migration. 

\subsection*{Invasion of rare genotypes and coexistence}\label{sec:invas}

In our model (section ``Model''), each genotype is defined by three traits: $b$, $c$ and $d$. To determine how these will evolve in a population where they are being modified by mutations, we need to know whether mutant lineages will grow (or decline) starting from low densities. In this section we discuss the behavior of rare genotypes predicted by Eq. \eqref{eq:master}. 

Suppose that a population with a single genotype $i$ is in equilibrium. Then $R_i=0$, $\overline{c}=c_i$ and $\Delta n_i = 0$, and so Eq. \eqref{eq:master} gives
\begin{equation}
b_i\left(e^{-L}+A_i\right)-d_i=0,\label{eq:equil}
\end{equation}
where $A_i=(1-(1+L)e^{-L})/L$. Now suppose that a new genotype $j$, which is initially rare, appears in the population. Then $A_j\ll R_j$, $l_j\approx 0$ and $\overline{c}\approx c_i$, and so, from Eq. \eqref{eq:master}, $n_j$ will increase if 
\begin{equation}
b_j \left(e^{-L}+R_j\frac{c_j}{c_i}\right)-d_j>0,\label{eq:invad}
\end{equation}
where $R_j\approx (1-e^{-L})/\left(\frac{c_j}{c_i}+\frac{L-1-e^{-L}}{1-(1+L)e^{-L}}\right)$. 

Combining Eqs. \eqref{eq:equil} and \eqref{eq:invad}, we see that $j$ will invade if it is superior in any one of the three traits, but is otherwise identical to $i$. If the new genotype has the same competitive ability $c_j=c_i$, then $R_j\approx A_i$ and Eqs. \eqref{eq:equil} and \eqref{eq:invad} imply that invasion occurs when $b_jd_i-b_id_j>0$, and in particular when $b_j>b_i$ with $d_j=d_i$, or when $d_j<d_i$ with $b_j=b_i$. In the case that the new genotype has a different competitive ability but the same $b_i$ and $d_i$, Eqs. \eqref{eq:equil} and \eqref{eq:invad} imply that invasion occurs when $R_j c_j/c_i > A_i$; it is not hard to verify that this occurs if and only if $c_j>c_i$ using the simplified expressions for $A_i$ and $R_j$ given after Eqs. \eqref{eq:equil} and \eqref{eq:invad} respectively. Moreover, if $j$ invades in any of these cases, it will eventually exclude $i$, since it is strictly superior. 

However, stable coexistence is possible between genotypes that are superior in different traits. To illustrate, suppose that $j$ is better at securing territories ($c_j>c_i$), that $i$ is better at producing propagules ($b_i>b_j$), and that $d_i=d_j$. Coexistence occurs if $j$ will invade an $i$-dominated population, but $i$ will also invade a $j$-dominated population (``mutual invasion''). It is not hard to show that this is possible, since if $b_i$ is so large that $L\gg 1$ when $i$ is dominant, and $b_j$ is so small that $L\ll 1$ when $j$ is dominant, then, combining Eqs. \eqref{eq:equil} and \eqref{eq:invad}, we find that $i$ invades $j$ because $b_i>b_j$, while $j$ invades $i$ provided that
\begin{equation}
b_jc_jR_j-b_i c_i A_i>0. \label{eq:jinvadcoex}
\end{equation}
Thus, coexistence occurs if $c_j/c_i$ is large enough. Intuitively, the mechanism for coexistence is that territorial contests are important in an $i$-dominated population (high $L$), ensuring that the $c$-specialist $j$ is not excluded, yet territorial contests are irrelevant in a $j$-dominated population (low $L$), ensuring that the $b$-specialist $i$ is not excluded. Fig.~\ref{fig:coex} shows an example of this coexistence between $b$ and $c$ specialists. 

A similar argument applies for coexistence between high-$c$ and low-$d$ specialists; again coexistence occurs because the importance of territorial contests declines along with propagule density $L$ as the $c$-specialist increases in frequency. Mutual invasibility is not possible between $b$- and $d$-specialists (although genotypes with exactly $c_i=c_j$ and $b_i/d_i=b_j/d_j$ do not exclude each other --- this follows from the fact that all propagules have the same probability of success when $c_i=c_j$ i.e. $A_i+R_i=A_j+R_j$).

If the rare genotype $j$ arises due to mutation, then its initial low-density behavior is more complicated than the above invasion analysis suggests. The mutant lineage starts with one individual $n_j=1$, and remains at low abundance for many generations after its initial appearance. During this period, the mutant abundance $n_j$ will behave stochastically, and the deterministic equations \eqref{eq:growthsumuncoupled} and \eqref{eq:master} do not apply (section ``Model''). However, if $n_j$ becomes large enough, its behavior will become effectively deterministic, and governed by Eq. \eqref{eq:master}. For mutants with fitness greater than the population mean fitness, this occurs when $n_j$ is of order $1/s$ \citep{desai_2007}, where the selection coefficient $s$ is the mutant's fitness advantage (i.e. $s=\frac{\Delta n_i/n_i}{\sum_i\Delta n_i/n_i\times n_i/N}-1$). Here we do not consider the initial stochastic behavior of novel mutants, and have restricted our attention to the earliest deterministic behavior of rare genotypes. In particular, for beneficial mutations we have only considered the case where $s$ is large enough that deterministic behavior starts when $n_j \ll N$.

\subsection*{Primary strategies and Grime's triangle}

We now discuss which changes in the traits $b, c$ and $d$ will be particularly favored under different environmental conditions. Of specific interest are Grime's ``disturbance'', ``stress'' and ``ideal'' environmental archetypes. To proceed, we need to map these verbal archetypes to quantitative parameter regimes in our model. 

The ideal environmental archetype is characterized by the near-absence of stress and disturbance. Consequently, $d_i\ll 1$, whereas $b_i$ is potentially much larger than $1$. From Eq. \eqref{eq:deltN}, the equilibrium value of $L$ only depends on the ratio of birth and death rates. For one genotype, $L/(1-e^{-L})=b_i/d_i$, and so the propagule density is high $L\approx b_i/d_i\gg 1$. Moreover, since $L=b_i\frac{N}{N-T}=b_i\frac{1}{1-T/N}$ by definition, population density is also high $N/T\approx 1$. Thus, almost every unoccupied territory will be heavily contested. 

The disturbance archetype is characterized by unavoidably high extrinsic mortality caused by physical destruction. Disturbances do not only affect adults as in Eq. \eqref{eq:delttot}, but also juveniles in the process of territorial contest. These juvenile deaths can be represented as a fractional reduction in the number of territories secured. To illustrate, we assume that the disturbance is equally damaging to adults and juveniles, so that only $(1-d_i)\Delta_+ n_i$ rather than $\Delta_+ n_i$ territories are secured by genotype $i$ each iteration. Then, the disturbance archetype is characterized by $d_i$ being close to $1$ for all genotypes (almost all adults and juveniles are killed each iteration). From Eq. \eqref{eq:deltN}, the single genotype equilibrium is given by $L/(1-e^{-L})=d_i/[(1-d_i)b_i]$, and since $L\ll 1$ and $N/T\ll 1$ due to high mortality, we have $L\approx 2(1-d_i/[(1-d_i)b_i])$. Clearly $b_i$ must be exceptionally large to ensure population persistence. The terms proportional to $c_i/\overline{c}$ in Eq. \eqref{eq:master} are then negligible, and $\Delta_+ n_i$ depends primarily on $b_i$. 

The stress archetype is more ambiguous, and has been the subject of an extensive debate in the plant ecology literature (the ``Grime-Tilman'' debate; \citealt{aerts_1999} and references therein). Stressful environments severely restrict growth and reproduction, so that $b\ll 1$ \cite{grime_1974,grime_1977}. Mutations which appreciably improve $b$ will be either non-existent or extremely unlikely, so $b$ is constrained to remain low. In Grime's view, under these conditions the rate at which propagules successfully develop to adulthood cannot appreciably exceed the mortality rate. This implies $b/d\approx 1$ in our model, and so the propagule density $L$ is suppressed to such low levels that there are essentially no territorial contests occurring. 

The alternative view is that, while stressful environments imply lower $b$ and support a lower number of individuals per unit area compared what is attainable in ideal environments, stressed populations are actually at high densities relative to the environmental carrying capacity, and are highly competitive \citep{taylor_1990}. In the particular case that stress is caused by scarcity of consumable resources, it is argued that we should expect intense resource competition (for empirical support, see \citealt{davis_1998}). Thus, $b$ may actually appreciably exceed $d$ under stressful conditions, even though the absolute value of $b$ is small. 

The mapping of environmental achetypes to our model parameters is summarized in the first two rows of Fig.~\ref{fig:table}. Also shown is the approximate dependence of $\Delta_+ n_i$ on $b_i$ and $c_i$ for each archetype (third row). These can be used infer the expected direction of evolution for the traits $b$, $c$ and $d$ (fourth row) as follows. 

As noted in the previous section, if beneficial mutations establish (i.e survive the low-abundance stochastic regime), they will proceed to grow deterministically according to Eq. \eqref{eq:master}. The probability of establishment increases with the mutant fitness advantage, and is therefore typically on the order of one percent, whereas the fixation of neutral mutations is exceedingly unlikely (probability of order $1/N$). Consequently, the direction of evolutionary change is determined by which trait changes are both available, and confer an appreciable benefit, where availability is subject to constraints imposed by the environment. 

For example, in Grime's version of the stress archetype, $L$ is low, so competition is not important, and only mutants with greater $b$ or lower $d$ will have an appreciably greater $\Delta n_i$. Mutations in $c$ are effectively neutral, and will rarely fix. However, by definition of the stress archetype, $b$ is constrained to be small. Thus, while some rare mutations may produce small improvements in $b$, it is much more likely that mutations will arise that lower $d$, making this the expected direction of evolutionary change for Grime's stress archetype. 

\begin{figure}
\centering
\begin{tabular}{l*{4}{c}}
  & Ideal & Disturbance* & Stress (G) & Stress (T) \\ \hline
  Constraints & $d \ll 1$ & $d \approx 1$ & $b \ll 1$ & $b \ll 1$ \\
  Other parameters & $b\gg d$ & $b\gg d$ & $b\approx d$ & $b>d$ \\
  Density $N/T$  & High & Low & Low & High \\
  $\Delta_+ n_i\propto$ & $b_i c_i$ & $b_i$ & $b_i$ & $b_i c_i$ \\
  Evolution for & $\uparrow b$, $ \uparrow c$ & $\uparrow b$, $\downarrow d$ & $\downarrow d$ & $\uparrow c, \downarrow d$
\end{tabular}
\caption{\label{fig:table} The realization of Grime's environmental archetypes in our model, as well as the low-$T$ variant of the stress archetype. Shown are the mapping to our parameters of each archetype, the approximate dependence of $\Delta_+ n_i$ on $b_i$ and $c_i$, as well as the corresponding expected evolutionary changes in $b_i$, $c_i$ and $d_i$. *Mortality affects both adults and juveniles in the disturbance archetype, with $\Delta_+ n_i$ replaced by $(1-d_i)\Delta_+ n_i$ in Eq. \eqref{eq:delttot}.}
\end{figure}

Following Grime's original argument for a triangular scheme \citep{grime_1977}, Fig.~\ref{fig:axes} represents each environmental archetype schematically as a vertex on a triangular space defined by perpendicular stress and disturbance axes. The ideal archetype lies at the origin (no stress or disturbance), while the stress and disturbance archetypes lie at the limits of survival on their respective axes. The hypotenuse connecting the stress and disturbance endpoints represents the limits of survival in the presence of a combination of stress and disturbance. The direction of evolutionary change is different at each vertex, leading to the emergence of different trait clusters or ``primary strategies''. 

\begin{figure}
\centering
\includegraphics[scale=1]{axes.pdf}
\caption{\label{fig:axes} The realization of Grime's triangle in our model. Schematic representation of the triangular space bounded by the low/high extremes of stress/disturbance. The low-$T$ interpretation of stress is also shown. The vertices of the triangles correspond to environmental archetypes. Selection favors different traits at each vertex, leading to different trait clusters.} 
\end{figure}

How does Fig.~4 compare to empirical analyses of Grime's C/S/R strategies? In our comparison we will stick to fishes, corals and plants, for which three-way primary strategy schemes are well developed \citep{grime_1977,winemiller_1992,darling_2012}. The connection of our model to fish strategies is necessarily more tentative, given that fishes are motile and not all territorial, and the starting assumption of our model is site-occupancy. 

In disturbed environments, we predict evolution for higher $b$ and lower $d$, but not higher $c$. Higher $b$ means higher fecundity, but not necessarily mass propagule production: $b$ represents only those propagules which sucessfully develop into juveniles in unoccupied territories. This is broadly consistent with the ruderal primary strategy. Plant ruderals devote a large proportion of their productivity to seed production \cite{grime_1977}, whereas the analogous ``opportunistic'' strategists in fishes have large intrinsic growth rates \citep{winemiller_1992}. In corals, a distinguishing feature of the ruderal cluster is brood spawning (rather than broadcast spawning). This corresponds to higher parental investment and lower overall propagule production, but potentially also higher $b$ at low densities, since broadcast spawners are vulnerable to a powerful Allee effect at the egg fertilization stage \citep{knowlton_2001}. Lower $d$ could be achieved by improved individual resistance to physical destruction, but it is hard to reduce mortality in the face of severe disturbances. Given this constraint, shortening the time to reproductive maturity (the iteration time in our model) is an effective way of reducing the chance of death per iteration $d$. An exceptionally short life cycle is probably the most defining characteristic of ruderals \citep{grime_1977,winemiller_1992,darling_2012}.

In stressful environments, we predict evolution for lower $d$, and also  for higher $c$ in the low-$T$ interpretation of the stress archetype. Lowering $d$ is obviously essential when $b\ll 1$, and stress tolerant plants and corals have long life spans, allowing for long intervals between successful recruitments (and episodic broadcast spawning in corals). For fishes, the ``equilibrium'' strategy is the analogue of Grime's stress tolerator. This strategy is associated with resource limitation, and is also characterized by long life span, as well as high parental investment in tiny broods. This may reflect a high-$c$ strategy in the face of intense competition for severely limited resources (the low-$T$ interpretation).

In ideal environments, we predict evolution for higher $b$ and $c$, but not lower $d$. In plants and corals, a key mechanism for winning territorial contests is rapidly outgrowing and ``shading out'' competitors; not surprisingly, rapid individual growth is a defining feature of the competitor trait cluster \citep{grime_1977,darling_2012}. Evolution for higher $b$ under high-density, competitive conditions may seem counter-intuitive. Neither particularly high nor low $b$ have been associated with the competitor strategy in plants and corals. However, for fishes, the analogous ``periodic'' strategy is characterized by enormous spawn sizes as well as rapid development \citep{winemiller_1992,winemiller_2015}, suggesting a strategy of ensuring that many propagules actually end up contesting areas favorable for development (higher $b$). The evolution of $b$ in ideal environments will be discussed further in the Discussion.

\section*{Discussion}

Unlike Grime's classic ternary plot \citep{grime_1974}, which represents anti-correlations between traits relevant for success in different environmental archetypes, our realization of Grime's triangle (Fig.~4) refers instead to the direction of adaptive trait evolution under different regimes of stress and disturbance. As discussed in section ``Primary strategies and Grime's triangle'', over time our predicted trait evolution should lead to trait values consistent with Grime's scheme. In making these predictions, we have made no reference to any kind of trade-offs or pleitropy, even though trade-offs are often invoked to explain primary strategy schemes \citep{macarthur_1967,winemiller_1992,aerts_1999}. Thus, while trade-offs may amplify specialization, they are not necessary for it. As an example of a trade-off, corals which rapidly out-shade neighbors have a tall, branched morphology which is vulnerable to disturbances, and so, all else being equal, ideal environment $c$-strategists will suffer higher mortality from disturbances. Fig.~\ref{fig:axes} gives the same conclusion without invoking trade-offs; mutations which reduce disturbance vulnerability are essentially neutral under ideal conditions, leading to no improvements in mortality from disturbances, whereas $c$ will tend to increase over time. 

Our prediction of evolution for higher $b$ in ideal environments is counter to the expectations of MacArthur's $r$/$K$ dichotomy \citep{macarthur_1967} since $b$ is closely related to the maximal, low-density growth rate $r=b-d$, and ideal environments support high population densities which should be subject to ``$K$-selection''. However, in the Introduction, we noted that the $r$-$K$ dichotomy is not consistent with empirical studies showing that maximal growth rate and saturation density (measured by abundance) are positively correlated, both between species/strains \citep{luckinbill_1979,kuno_1991,hendriks_2005,fitzsimmons_2010}, and as a result of experimental evolution \citep{luckinbill_1978,luckinbill_1979}. From the perspective of our model, these correlations are not surprising since the saturation density, which is determined by a balance between births and deaths, increases with $b$. Our higher-$b$ prediction simply reflects the fact that, all else being equal, producing more propagules is always advantageous, regardless of population density, a fact lost in the simple logistic interpretation of the $r$/$K$ scheme. 

Confusingly, the term ``$K$-selection'' has sometimes been used to refer generally to selection at high density; this encompasses both selection for higher saturation density --- ``$K$'' in the logistic equation --- as well as selection for competitive ability. To avoid this ambiguity, the latter form of selection has been called ``$\alpha$-selection'' after the competition coefficients in the Lotka-Volterra equation \citep{gill_1974,case_1974,joshi_2001}. Unlike saturation density, there is support for a negative relationship between competitive success at high densities and maximal growth rate \citep{luckinbill_1979}; this could be driven by a tradeoff between individual size and reproductive rate. However, competitive success as measured by $\alpha$ (i.e. the per-capita effect of one genotype on another genotype's growth rate) is only partly determined by individual competitive ability --- in the presence of age-structure and territoriality, it also includes the ability of each genotype to produce contestants i.e. $b$ in our model. In contrast, our $c$ is strictly competitive ability only --- as such, changes in $c$ do not directly affect population density (section ``Model''), nor the ability to produce contestants.

$K$-selection in the sense of selection for a greater environmental carrying capacity for given birth and death rates, sometimes referred to as ``efficiency'' \citep{macarthur_1967}, would be represented in our model by smaller individual territorial requirements. To a first approximation, two co-occurring genotypes which differ by a small amount in their territorial requirements only should have the same fitness since the costs or benefits of a change in the amount of unocupied territory is shared equally among genotypes via the propagule density $L$. The situation is more complicated if those genotypes differ in multiple traits, and when the differences in territorial requirements become large enough that territorial contests can occur on different scales. We leave these complications for future work. 

The importance of $b$ in securing territories is a general feature of lottery competition. Indeed, as can be seen from Eq. \eqref{eq:lottery}, in the classic lottery model $b_i$ and $c_i$ are essentially equivalent in that only the products $b_i c_i$ matter \citep{chesson_1981}. This is no longer the case in our density- and frequency-dependent generalization of the classic lottery model, where stable co-existence is possible between $b$ and $c$ strategists. Given that the classic lottery model was specifically developed for studying species co-existence questions, this may seem surprising, but the focus in that case was on the role of environmental fluctuations in promoting co-existence rather than coexistence in a single, stable environment \citep{chesson_1981}. It is not clear whether correctly accounting for the behavior of species with low density would significantly alter the conclusions of \cite{chesson_1981}, but, in any case, species co-existence questions are beyond the scope of this manuscript. 

Rather, while our model can be applied to questions at an inter-species level (e.g. ecological invasions), our focus here is on the evolution of genotype frequencies within a population. Our ability to describe evolutionary processes is only possible because the model accounts for the growth of mutants from low densities. Given this focus, our assumption that there are no large $c$ discrepancies (section ``Mean field approximation'') amounts to a restriction on the amount of genetic variation in $c$ that will be sustained in the population. Since beneficial mutation effect sizes will typically not be much larger than a few percent, large $c$ discrepancies can only arise if the mutation rate is extremely large, and so the assumption will not be violated in most cases. However, this restriction could become important when looking at species interactions rather than genotype evolution.

In Fig. \ref{fig:table2} we compare our model with some of the other models and schemes touched upon here. In a sense this comparison is unfair: for instance, Grime's scheme was developed for an entirely different purpose (species classification by traits). As such, Fig. \ref{fig:table2} is not exhaustive and should be read more as a summary of our model's purpose. Like MacArthur's $r$/$K$ scheme, our model is motivated by the need to expand the treatment of selection in population genetics \citep{macarthur_1962}, i.e. to incorporate crucial ecological factors in our most genetically realistic models of evolution. Thus, viewing evolutionary ecology \citep{kokko_2007,pelletier_2009,schoener_2011} as a spectrum ranging from evolution-only to ecology-only, our model lies close to the understudied evolution-only end of the spectrum. By comparison, more familiar approaches to evolutionary ecology such as adaptive dynamics --- essentially ecology coupled with mutant invasion \citep{diekmann_2004} ---  lie close to the ecology-only end of the spectrum. 

\begin{figure}
\centering
\begin{tabularx}{\linewidth}{lXXXXX}
&Formal model?  & Ecologically meaningful? & Empirically-grounded trait scheme? & Generality beyond specific scenarios? & Genetically flexible? \\ \hline
Density-dependent lottery & \ding{51} & \ding{51} & \ding{51} & \ding{51} & \ding{51} \\  
  MacArthur's $r$/$K$ + $\alpha$ & \ding{51} & \ding{51} & \ding{55}* & \ding{51} & \ding{51} \\
  Grime's C/S/R & \ding{55} & \ding{51} & \ding{51} & \ding{51} & NA \\
  Traditional pop. gen. & \ding{51} & \ding{55} & \ding{55} & \ding{51} & \ding{51} \\
  Eco-evo:  \\
  \quad Adaptive dynamics & \ding{51} & \ding{51} & \ding{51} & \ding{51}** & \ding{55} \\
  \quad Brute force simulation & \ding{51} & \ding{51} & NA & \ding{55} & \ding{51}
\end{tabularx}
\caption{\label{fig:table2} Comparison of our density-dependent lottery model with related models and schemes in ecology and evolutionary biology. *MacArthur's $r$- and $K$-, as well as $\alpha$ selection, were all derived theoretically. Applications to traits came later, and with mixed success (see ``Introduction'' and ``Discussion''). **In practice, most of the adaptive dynamics literature focuses on specific eco-evolutionary outcomes such as evolutionary ``branching'' \citep{geritz_1997} or mutualisms \citep{ferriere_2002}, but in principle it can be applied with any fitness model including our density-dependent lottery.}
\end{figure}

In our view, our model has two major limitations as a general-purpose model of density-dependent selection: a reliance on interference competition for durable resources (territoriality), and the restriction of competition to juveniles (lottery recruitment to adulthood). In some respects this is the complement of resource competition models, which restrict their attention to exploitation competition, typically without age structure \citep{tilman_1982}. In the particular case that resources are spatially localized (e.g. due to restricted movement through soils), then resource competition and territorial acquisition effectively coincide, and in principle resource competition could be represented by a competitive ability $c$ (or conversely, $c$ should be derivable from resource competition). The situation is more complicated if the resources are well-mixed, since, in general, resource levels then need to be explicitly tracked. It seems plausible that explicit resource tracking is not necessary when the focus is on the evolution of similar genotypes rather than the stable co-existence of widely differing species. We are not aware of any attempts to delineate conditions under which explicit resource tracking is unnecessary even if it is assumed that community structure is ultimately determined by competition for consumable resources. More work is needed connecting resource competition models to the density-dependent selection literature, since most of the existing work is focused on the narrower Grime-Tilman-debate issue of resolving the role of competition at different levels of resource availability \citep{aerts_1999,davis_1998,tilman_2007}. 

On the other hand, our model does remarkably well in capturing apparently general features of selection under different environmental conditions using only three trait parameters: $b$, $c$ and $d$. The clean separation of a strictly-relative $c$ parameter is particularly useful from an evolutionary genetics perspective, essentially embedding a zero-sum fitness trait within a non-zero-sum fitness model. This could have interesting applications for modeling the impacts of intra-specific competition on species extinction, for example due to clonal interference \citep{gerrish_1998,desai_2007} between $c$-strategists on the one hand, and $b$- and $d$- strategists on the other. 


\bibliographystyle{amnatnat}
\bibliography{reference} 

\section*{Appendix A: Poisson approximation}

For each genotype's dispersal, the counts of propagules across unnocupied territories follows a multinomial distribution with equal probabilities of landing in each territory. Thus, the $x_i$ in different territories are not independent random variables. However, for sufficiently large $T$, holding $n_i/T$ fixed, the Poisson limit theorem implies that this multinomial distribution for the $x_i$ accross territories is closely approximated by a product of independent Poisson distributions for each territory, each with rate parameter $l_i$. Here we have used the fact that large $T$ implies large $U$ except in the biologically uninteresting case that there is vanishing population turnover $d_i \sim 1/T$. 

Under the Poisson approximation, the total number of genotype $i$ propagules $\sum x_i$ (summed over unoccupied territories) will deviate about its mean value $m_i$. Since the coefficient of variation of $\sum x_i$ is proportional to $1/\sqrt{m_i}$, these fluctuations are negligible unless $m_i$ is very small (say of order $10^2$ or less). These fluctuations in $m_i$ could be regarded as a feature rather than a flaw since having $m_i$ be exactly constant per generation (for given $b_i$ and $n_i$) is biologically unrealistic. In the canonical model of genetic drift, the Wright-Fisher model, the number of offspring per genotype fluctuates from generation to generation and is approximately Poisson distributed. Nevertheless, for simplicity and ease of comparison with the classic lottery model, we ignore fluctuations in $m_i$ and only account for Poisson fluctuations in the number of propagules landing in each territory. 

\section*{Appendix B: Derivation of growth equation}

We separate the right hand side of Eq.~\eqref{eq:growthsumuncoupled} into three components $\Delta_+ n_i = \Delta_u n_i+\Delta_r n_i+\Delta_a n_i$ which vary in relative magnitude depending on the propagule densities $l_i$. Following the notation in the main text, the Poisson distributions for the $x_i$ (or some subset of the $x_i$) will be denoted $p$, and we use $P$ as a general shorthand for the probability of particular outcomes.

\subsection*{Growth without competition}

The first component, $\Delta_u n_i$, accounts for territories where only one focal propagule is present $x_i=1$ and $x_j=0$ for $j\neq i$ ($u$ stands for ``uncontested''). The proportion of territories where this occurs is $l_i e^{-L}$, and so 
\begin{equation}
\Delta_u n_i=Ul_i e^{-L}=m_i e^{-L}.
\end{equation}

\subsection*{Competition when rare}

The second component, $\Delta_r n_i$, accounts for territories where a single focal propagule is present along with at least one non-focal propagule ($r$ stands for ``rare'') i.e. $x_i=1$ and $X_i\geq 1$ where $X_i=\sum_{j\neq i} x_j$ is the number of nonfocal propagules. The number of territories where this occurs is $Up_i(1)P(X_i\geq 1)=b_i n_i e^{-l_i}(1-e^{-(L-l_i)})$. Thus 
\begin{equation}
\Delta_r n_i = m_i e^{-l_i}(1-e^{-(L-l_i)})\left\langle  \frac{c_i}{c_i +\sum_{j\neq i} c_j x_j } \right\rangle_{\tilde{p}},  \label{eq:deltr}
\end{equation}
where $\langle \rangle_{\tilde{p}}$ denotes the expectation with respect to $\tilde{p}$, and $\tilde{p}$ is the probability distribution of nonfocal propagule abundances $x_j$ \textit{after} dispersal, in those territories where exactly one focal propagule, and at least one non-focal propagule, landed. 

We will show that, with respect to $\tilde{p}$, the standard deviation $\sigma_{\tilde{p}}(\sum_{j\neq i} c_j x_j)$, is much smaller than $\langle\sum_{j\neq i} c_j x_j\rangle_{\tilde{p}}$. Then $x_j$ can be replaced by its mean in the last term in Eq.~\eqref{eq:deltr},
\begin{equation}
\left\langle\frac{c_i}{c_i +\sum_{j\neq i} c_j x_j}\right\rangle_{\tilde{p}}\approx \frac{c_i}{c_i +\sum_{j\neq i} c_j \langle x_j\rangle_{\tilde{p}}}.\label{eq:meanfieldr}
\end{equation}

We first calculate $\langle x_j \rangle_{\tilde{p}}$. Let $X=\sum_j x_j$ denote the total number of propagules in a territory and ${\mathbf x_i}=(x_1,\ldots,x_{i-1},x_{i+1}\ldots,x_G)$ denote the vector of non-focal abundances, so that $p({\mathbf x_i})=p_1(x_1)\ldots p_{i-1}(x_{i-1})p_{i+1}(x_{i+1})\ldots p_G(x_G)$. Then, $\tilde{p}$ can be written as
\begin{align}
\tilde{p}({\mathbf x_i})&=p({\mathbf x_i}|X\geq 2,x_i=1)\nonumber\\
&=\frac{P({\mathbf x_i},X\geq 2|x_i=1)}{P(X\geq 2)}\nonumber\\
&=\frac{1}{1-(1+L)e^{-L}}\sum_{X=2}^{\infty} P(X) p({\mathbf x_i}|X_i=X-1),
\end{align}
and so
\begin{align}
\langle x_j \rangle_{\tilde{p}}&=\sum_{\mathbf x_i} \tilde{p}({\mathbf x_i})x_j\nonumber\\
&=\frac{1}{1-(1+L)e^{-L}}\sum_{X=2}^{\infty} P(X) \sum_{\mathbf x_i} p({\mathbf x_i}|X_i=X-1)x_j.
\label{eq:raremonster1}
\end{align}
The inner sum over ${\mathbf x_i}$ is the mean number of propagules of a given nonfocal type $j$ that will be found in a territory which received $X-1$ nonfocal propagules in total, which is equal to $\frac{l_j}{L-l_i}(X-1)$. Thus, 
\begin{align}
\langle x_j \rangle_{\tilde{p}}&=\frac{l_j}{1-(1+L)e^{-L}}\frac{1}{L-l_i}\sum_{k=2}^{\infty} P(X) (X-1)\nonumber\\
&=\frac{l_j}{1-(1+L)e^{-L}}\frac{L-1+e^{-L}}{L-l_i},
\label{eq:meanxjrare}
\end{align}
where the last line follows from $\sum_{X=2}^{\infty} P(X)(X-1)=\sum_{X=1}^{\infty} P(X)(X-1)=\sum_{X=1}^{\infty} P(X)X-\sum_{X=1}^{\infty}P(X)$.

For analyzing the relative fluctuations in $\sum_{j\neq i} c_j x_j$, Eq. \eqref{eq:meanxjrare} is unnecessarily complicated. We instead use the following approximation. Rather than evaluating the situation in each territory after dispersal as above, we replace $\tilde{p}$ by $\tilde{q}$, defined as the ${\mathbf x_i}$ Poisson dispersal probabilities conditional on $X_i\geq1$, independently of the outcome of $x_i$. This gives $\langle x_j \rangle_{\tilde{q}}=\langle x_j \rangle_p/C=l_j/C$, 
\begin{align}
\sigma_{\tilde{q}}^2(x_j)&=\langle x_j^2 \rangle_{\tilde{q}}-\langle x_j \rangle_{\tilde{q}}^2\nonumber\\
&=\frac{1}{C}\langle x_j^2 \rangle_p-\frac{l_j^2}{C^2}\nonumber \\
&=\frac{1}{C}(l_j^2 + l_j)-\frac{l_j^2}{C^2}\nonumber \\
&=\frac{l_j^2}{C}\left(1-\frac{1}{C}\right)+\frac{l_j}{C},\label{eq:varr}
\end{align}
and 
\begin{align}
\sigma_{\tilde{q}}(x_j,x_k)&=\langle x_j x_k \rangle_{\tilde{q}}-\langle x_j \rangle_{\tilde{q}}\langle x_k \rangle_{\tilde{q}}\nonumber\\
&=\frac{1}{C}\langle x_j x_k \rangle_p-\frac{l_jl_k}{C^2}\nonumber\\
&=\frac{l_j l_k}{C}\left(1-\frac{1}{C}\right),\label{eq:covr}
\end{align}
where $C=1-e^{-(L-l_i)}$ and $j\neq k$. The distribution $\tilde{q}$ only approximates the situation after dispersal, since knowing that one focal genotype is among the propagules present restricts the possible outcomes for the $x_j$, so that the $x_j$ cannot strictly be treated as independent of $x_i$. This seemingly minor distinction has meaningful consequences. To illustrate, suppose that the focal genotype is rare and the propagule density is high ($l_j \approx L\gg 1$). Then Eq. \eqref{eq:meanxjrare} correctly predicts that there are on average $L-1$ nonfocal propagules $\langle x_j \rangle_{\tilde{p}}\approx L-1$, with the focal propagule correctly excluded, whereas $\tilde{q}$ predicts one extra $\langle x_j \rangle_{\tilde{q}}\approx L$. As a result, $\tilde{q}$ gives pathological behavior for rare invaders (they have a rarity disadvantage), but its moments are quantitatively similar enough to those of $\tilde{p}$ that it is sufficient for analyzing the relative fluctuations in $\sum_{j\neq i} c_j x_j$.

Decomposing the variance in $\sum_{j\neq i} c_j x_j$,
\begin{equation}
\sigma_{\tilde{q}}^2(\sum_{j\neq i} c_j x_j)=\sum_{j\neq i}\left[c_j^2\sigma_{\tilde{q}}^2(x_j)+2\sum_{k>j}c_j c_k\sigma_{\tilde{q}}(x_j,x_k)\right],\label{eq:vartotr}
\end{equation}
and using the fact that $\sigma_{\tilde{q}}(x_j,x_k)$ and the first term in Eq. \eqref{eq:varr} are negative because $C<1$, we obtain an upper bound on the relative fluctuations in $\sum_{j\neq i} c_j x_j$, 
\begin{equation}
\frac{\sigma(\sum_{j\neq i} c_j x_j)}{\langle\sum_{j\neq i} c_j x_j\rangle}<C^{1/2}\frac{\left(\sum_{j\neq i}c_j^2 l_j\right)^{1/2}}{\sum_{j\neq i}c_j l_j}. \label{eq:cvr}
\end{equation}

Without loss of generality, we restrict attention to the case that the total nonfocal density $L-l_i$ is of order $1$ or larger (otherwise $\Delta_r n_i$ does not contribute significantly to $\Delta_+ n_i$ because $\Delta_r n_i$ is proportional to $C=1-e^{-(L-l_i)}$).

Then, when at least some of the nonfocal propagule densities are large $l_j\gg 1$, the RHS of Eq.~\eqref{eq:cvr} is $\ll 1$, as desired. This is also the case if none of the nonfocal genotype densities are large and the $c_j$ are all of similar magnitude (their ratios are of order one); the worst case scenario occurs when $L-l_i\sim O(1)$, in which case the negative covariances (Eq.~\eqref{eq:covr}) which were neglected in the RHS of Eq.~\eqref{eq:cvr} significantly reduce the overall variance $\sigma_{\tilde{q}}^2(\sum_{j\neq i} c_j x_j)$.

However, the relative fluctuations in $\sum_{j\neq i} c_j x_j$ can be large if some of the $c_j$ are much larger than the others. Specifically, in the presence of a rare, extremely strong competitor ($c_j l_j\gg c_{j'} l_{j'}$ for all other nonfocal genotypes $j'$, and $l_j\ll 1$), then the RHS of Eq. \eqref{eq:cvr} can be large and we cannot make the replacement Eq.~\eqref{eq:meanfieldr}. 

Substituting Eqs. \eqref{eq:meanfieldr} and \eqref{eq:meanxjrare} into Eq.~\eqref{eq:deltr}, we obtain
\begin{equation}
\Delta_r n_i\approx m_i R_i\frac{c_i}{\overline{c}}, \label{eq:deltrfinal}
\end{equation}
where $R_i$ is defined in Eq.~\eqref{eq:Dr}.

\subsection*{Competition when abundant}

The final contribution, $\Delta_a n_i$, accounts for territories where two or more focal propagules are present ($a$ stands for ``abundant"). Similarly to Eq.~\eqref{eq:deltr}, we have 
\begin{equation}
\Delta_a n_i=U(1-(1+l_i)e^{l_i})\left\langle \frac{c_i x_i}{\sum_j c_j x_j} \right\rangle_{\hat{p}}\label{eq:delta}
\end{equation}
where $\hat{p}$ is the probability distribution of both focal and nonfocal propagaule abundances \textit{after} dispersal in those territories where at least two focal propagules landed. 

Again, we show that the relative fluctuations in $\sum c_j x_j$ are much smaller than $1$ (with respect to $\hat{p}$), so that,
\begin{equation}
\left\langle \frac{c_i x_i}{\sum_j c_j x_j} \right\rangle_{\hat{p}}\approx  \frac{c_i \langle x_i \rangle_{\hat{p}}}{\sum_j c_j \langle x_j\rangle_{\hat{p}}}.\label{eq:meanfielda}
\end{equation}
Following a similar procedure as for $\Delta_r n_i$, where the vector of propagule abundances is denoted ${\mathbf x}$, the mean focal genotype abundance is, 
\begin{align}
\langle x_i \rangle_{\hat{p}}&=\sum_{\mathbf x} x_i p(\mathbf x|x_i\geq 2)\nonumber \\
&=\sum_{x_i} x_i p(x_i|x_i\geq 2) \nonumber\\
&=\frac{1}{1-(1+l_i)e^{-l_i}}\sum_{x_i\geq 2} p(x_i)x_i\nonumber\\
&=l_i\frac{1-e^{-l_i}}{1-(1+l_i)e^{-l_i}}.
\end{align}
For nonfocal genotypes $j\neq i$, we have
\begin{align}
\langle x_j \rangle_{\hat{p}}&=\sum_{\mathbf x} x_j p(\mathbf x|x_i\geq 2)\nonumber \\
&=\sum_{X}P(X|x_i\geq 2)\sum_{\mathbf x} x_j p({\mathbf x}|x_i\geq 2,X)\nonumber\\
&=\sum_{X}P(X|x_i\geq 2)\sum_{x_i} p(x_i|x_i\geq 2,X) \sum_{\mathbf x_i} x_j p(\mathbf x_i|X_i=X-x_i)\nonumber\\
&=\sum_{X}P(X|x_i\geq 2)\sum_{x_i}p(x_i|x_i\geq 2,X) \frac{l_j(X-x_i)}{L-l_i} \nonumber\\
&=\frac{l_j}{L-l_i}\left[\sum_{X}P(X|x_i\geq 2)X - \sum_{x_i}p(x_i|x_i\geq 2) x_i \right]\nonumber\\
&=\frac{l_j}{L-l_i}\left( L\frac{1-e^{-L}}{1-(1+L)e^{-L}}- l_i\frac{1-e^{-l_i}}{1-(1+l_i)e^{-l_i}}\right). 
\end{align}

To calculate the relative fluctuations in $\sum_{j\neq i} c_j x_j$, we use a similar approximation as for $\Delta_r n_i$: $\hat{p}$ is approximated by $\hat{q}$, defined as the ${\mathbf x}$ dispersal probabilities in a territory conditional on $x_i>2$ (that is, treating the $x_j$ as independent of $x_i$). All covariances between nonfocal genotypes are now zero, so that $\sigma^2(\sum c_j x_j)=\sum c_j^2 \sigma^2(x_j)$, where $\sigma^2(x_j)=l_j$ for $j\neq i$. The expression for $\sigma^2(x_i)$ is more complicated, but in the relevant regime where $p(x_i=0)\approx 0$ (since otherwise $D\gg 1$ and $\Delta n_a$ is negligible), then
\begin{equation}
\sigma_{\hat{q}}^2(x_i)\approx\frac{l_i^2}{D}\left(1-\frac{1}{D}\right)+\frac{l_i}{D},
\end{equation}
where $D= 1-(1+l_i)e^{-l_i}$, analogous to Eq.~\eqref{eq:varr}, and 
\begin{equation}
\frac{\sigma_{\hat{q}}(\sum c_j x_j)}{\langle\sum c_j x_j\rangle} \approx\frac{\left(\sum_{j\neq i} c_j^2 l_j + c_i^2 \sigma_{\hat{q}}^2(x_i)\right)^{1/2}}{\sum_{j\neq i} c_j l_j + c_i l_i/D} \label{eq:cva}.
\end{equation}

Similarly to Eq.~\eqref{eq:cvr}, the RHS of \eqref{eq:cva} will not be $\ll 1$ in the presence of a rare, extremely strong competitor. When this is not the case, then since $l_i$ must be of order $1$ or larger for $\Delta_a n$ to make an appreciable contribution to $\Delta_+ n_i$, the RHS of Eq.~\eqref{eq:cva} is $\ll 1$ as desired. 

Combining Eqs. \eqref{eq:delta} and \eqref{eq:meanfielda}, we obtain
\begin{equation}
\Delta_a n_i=m_i A_i \frac{c_i}{\overline{c}},
\end{equation}
where $A_i$ is defined in Eq.~\eqref{eq:Da}.

%The logistic equation for a single genotype or species is given by
%\begin{equation}
%\frac{dN}{dt}=r\left(1-\frac{N}{K}\right)N \label{eq:logistic}
%\end{equation}
%where $N$ is the total population abundance or biomass, $r$ is the growth rate at  low densities, and the ``carrying capacity" $K$ is the value of $N$ where growth ceases.
%
%a simple, ubiquitous model of density-dependent population growth
%declines linearly with density,
%
%Eq.~\eqref{eq:logistic} is often interpreted as a linear approximation of a more complicated growth equation
%\begin{equation}
%\frac{dN}{dt}=r(N)N,
%\end{equation}
%where the growth rate $r(N)$ is some unknown, possibly complicated, function of $N$. 

%\subsection{Logistic growth and classic nesting sites model}

%In the nesting sites model, it is assumed that propagules landing on occupied territory do not survive, while those landing on unoccupied territory immediately occupy the territory. Thus, the probability that a propagule survives is simply $(1-N/K)$, where $N=\sum_i n_i$ is the total population size. Consequently, $n_i$ increases by $1$ over the interval $[t,t+\Delta t]$ with probability $b_i n_i (1-N/K)\Delta t$. Provided that the abundances $n_i$ are large enough that demographic stochasticity is negligible (i.e. they have established), we can treat $n_i(t)$ as a continuous variable with growth given by the expectation of $b_i n_i (1-N/K)\Delta t$, and so, taking $\Delta t \rightarrow 0$ we obtain
%\begin{equation}
%\frac{dn_i}{dt}=b_i\left(1-\frac{N}{K}\right)n_i. \label{eq:logistic}
%\end{equation}

%\section{Territorial availability}
%
%In Sec. \ref{sec:c} we assumed that adults have the same territorial requirement, regardless of genotype. This can be generalized to allow genotypes to differ in their territorial requirements. Each adult from genotype $i$ now requires $t_i$ units of territory, $T$ is the total territory available to the population, and $T-\sum_i t_i$ units of territory are unoccupied. 
%
%Competition between propagules with different territorial requirements is potentially much more complicated than the model in Sec. \ref{sec:c}, because competition is no longer neatly partitioned into a set of identical territories.  The larger a propagule's territorial requirement, the more neighboring propagules it will interact with during its development to adulthood. 
%
%
%A simple way to avoid these complications is to partition the unoccupied territory into uniform units of territory with size given by the largest territorial requirement $t^*={\rm max}_i t_i$. The uniform territory approach used in Sec.  \ref{sec:c}
%
%The biological intuition behind this 
%
%Thus $U=(T-\sum_i t_i n_i)/t^*$.


\end{document}
