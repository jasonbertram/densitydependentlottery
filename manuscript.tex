\documentclass[11pt]{article}
\usepackage[sc]{mathpazo}
\usepackage{amsmath,pifont}
\usepackage{fullpage}
\usepackage[authoryear,sectionbib,sort]{natbib}
\linespread{1.7}
\usepackage[utf8]{inputenc}
\usepackage{lineno}

\usepackage{graphicx} 
\usepackage{tabularx,setspace}

\title{Density-dependent selection and the limits of relative fitness}
%\author{Jason Bertram $^{1,\ast}$ \\ 
%Joanna Masel $^{1}$}

\date{}

\begin{document}

\maketitle

%\noindent{}1. Department of Ecology and Evolutionary Biology, University of Arizona, Tucson, AZ 85721.

%\noindent{}$\ast$ Corresponding author; e-mail: jbertram@email.arizona.edu.


\bigskip

%\textit{Manuscript elements}: 

\bigskip

\textit{Keywords}: $r$/$K$ selection, absolute fitness, eco-evo, competition-colonization trade-off, fluctuating selection, storage effect.

\bigskip


\linenumbers{}
\modulolinenumbers[1]

\newpage{}

\section*{Abstract}


new model of density-dependent selection by generalizing the fixed-density classic lottery model of territorial acquisition to accommodate arbitrary population densities. 

We show that, with density dependence, co-existence is possible in the lottery model in a stable environment. 

Inspired by natural \textit{Drosophila} populations, we consider co-existence under strong, seasonally-fluctuating selection coupled to large cycles in population density, and show that co-existence (stable polymorphism) is promoted via a combination of the classic storage effect and density-regulated population growth. 

\newpage{}


\section*{Introduction}

There are a variety of different measures of fitness. Some widely used examples in evolutionary ecology are expected lifetime reproductive ratio $R_0$, intrinsic growth rate $r$, saturation population density (often labeled ``$K$'') \citep{benton_2000}, and invasion fitness \citep{metz_1992}. In addition, ``relative fitness'' is the standard in much of evolutionary biology, particularly evolutionary genetics, where attention is generally restricted to relative genotypic proportions \cite[pp. 468]{barton_2007}. The variety of fitness measures is not problematic in itself, because different measures  may be more useful in different circumstances. But it should be clear how the measure being used is connected to the processes of birth and death which govern population biology \citep{metcalf_2007,doebeli_2017}. While such a connection is fairly clear for absolute fitness measures like $r$, relative fitness seems largely divorced from population biology. It has even been proposed that relative fitness be justified from more abstract measure-theoretical arguments, abandoning population biology altogether \citep{wagner_2010}.

In uncrowded populations, relative fitness simply represents differences in intrinsic growth rate. In continuous time, we have the canonical selection equation $\frac{d p_i}{dt}=(r_i-\overline{r})p_i$, where $r_i$ is the intrinsic exponential growth rate of type $i$, $p_i$ is its frequency, and $\overline{r}=\sum_i r_i p_i$ is the population mean $r$ \citep[pp. 26]{crow_1970}. If there are two types present, a wildtype and a mutant for instance, then Eq.~\eqref{eq:canonical} can be written as 
\begin{equation}
\frac{d p_i}{dt}=s p_i(1-p_i), \label{eq:canonical}
\end{equation}
where the constant selection coefficient $s$ is the difference in $r$ between types. The corresponding adaptive sweeps follow a logistic curve. 

The difficulty with Eq.~\eqref{eq:canonical} arises in crowded populations. It is then commonly assumed that total population density $N$ has reached its equilibrium value, which is assumed to be a fixed constant. This is exemplified by the Wright-Fisher model. Yet MacArthur famously showed that when population growth is density-regulated, selection in crowded populations is intimately connected to the ability to keep growing at higher densities than other types can tolerate \citep{macarthur_1967}. The classic example is the logistic model, where the type with the greatest saturation population density ``$K$'' excludes the others (Fig.~\ref{fig:Ksel}a). Similarly, the ``$R^*$ rule'', a central tenet of resource competition theory, states that when growth is limited by a single homogeneous consumable resource, the type able to deplete the resource to the lowest equilibrium density $R^*$ excludes the others \citep{grover_1997}. Differences in $R^*$ will often entail differences in saturation density. The Lotka-Volterra competition model also couples selection in crowded populations to changes in $N$ except in a few special cases (we return to this in section ``$K$-selection, $c$-selection and relative fitness''). In these examples, both $N$ and $s$ change during, and as a result of, adaptive sweeps. It would therefore seem that the ubiquitous constant-$N$, relative fitness description of selection is incompatible with a huge class of population ecological processes occurring in nature and experiments (Fig.~\ref{fig:Ksel}b).

\begin{figure}
\centering
\includegraphics[scale=0.8]{Kplot.pdf}
\caption{\label{fig:Ksel} Selection in crowded environments shown as a phase diagram for the densities of two types $n_1$ and $n_2$. (a) The logistic model $\frac{dn_1}{dt}=r_1(1-\frac{n_1+n_2}{K_1})n_1$ and $\frac{dn_2}{dt}=r_2(1-\frac{n_1+n_2}{K_2})n_1$ with $r_1=r_2$ and $K_1>K_2$. (b) The constant-$N$, relative fitness description of selection.}
\end{figure}

The relative fitness description has been justified in broadly two different ways for crowded populations (we do not address Wagner's [\citeyear{wagner_2010}] measure theorical justification, which explicitly seeks to be independent of population biology and thus falls outside of our scope). First, if we only care about type frequencies, we can sometimes simply ignore $N$ \cite[pp. 468]{barton_2007}, because Eq.~\eqref{eq:canonical} is unaffected by changes in $N$ if the mechanism regulating density applies equally to all types \citep{prout_1980}. While this does allow us to relax the restrictive assumption of constant $N$, it does not address the assumption of constant $s$. In the examples from the previous paragraph, different types are affected by the regulation of density in different ways; indeed, the type-specific responses to density  are at the center of MacArthur's argument and the literature on density-dependent selection that grew out of it. 

Second, constant $N$ and $s$ can both be seen as a short-term linear approximation \cite[pp. 277]{ewens_2004}. That is, within a sufficiently short time frame, $N$ and $s$ can be treated as constant. Provided that selection is sufficiently weak and stable over time, this ``short-term''  assumption is not a major restriction. Yet it is increasingly recognized that selection is not always weak, that it can fluctuate considerably over time, and that $N$ can vary by orders of magnitude over a few generations as a routine feature of a population's ecology \citep{messer_2016}. These are not rare exceptions, but occur widely in nature and the lab, including in wild \textit{Drosophila} \citep{bergland_14}. Nevertheless, relative fitness models like Wright-Fisher are the foundation for much of the population genetic literature, and are still widely used without considering the ``short-term'' restriction or the gulf with population ecology \citep{mallet_2012}. Thus, it is important to understand the population ecological basis of relative fitness models, both to gain insight into their domain of applicability, and as part of the broader challenge of synthesizing ecology and evolution.

Here we analyze the population ecology of relative fitness using a novel model of density-dependent population growth based on territorial contests. Rather than starting from the standard models of population growth mentioned above, our point of departure  is the classic lottery model \citep{sale_77,chesson_1981}. Like the Wright-Fisher model, the classic lottery assumes a saturated population with constant $N$, and fitness involves a product of fertility and juvenile viability \citep[pp. 185]{crow_1970}. Unlike the Wright-Fisher model, generations can overlap in the lottery model. We generalize the lottery model to allow population density to vary, and show that this model can be interpreted as a density-dependent generalization of the Wright-Fisher model with overlapping generations.

We show that when this model reaches a demographic steady-state, the constant-$N$, relative fitness picture emerges. Futhermore, we show that our model is entirely consistent with MacArthur's analysis of selection in crowded populations. In particular, we emphasize that MacArthur's argument does not justify the widespread intuition that selection in crowded environments is necessarily connected to achieving greater densities. This is largely an artifact of the models historically used in the density-dependent selection literature, which ignore relative contests. 

 Our first task is to analytically extend the classic lottery model to correctly account for low density behavior (sections ``Model'' and ``Mean field approximation''). We then...

 
\section*{Model}\label{sec:model}

\subsection*{Model assumptions and definitions} 

We assume that reproductively mature individuals (``adults'') each require their own territory to survive and reproduce (Fig.~\ref{fig:lottery}). All territories are identical, and the total number of territories is $T$. Time $t$ advances in discrete iterations, each representing the time from birth to reproductive maturity. In iteration $t$, the number of adults of the $i$'th type is $n_i(t)$, the total number of adults is $N(t)=\sum_i n_i(t)$, and the number of unoccupied territories is $U(t)=T-N(t)$. 

We assume that the $n_i$ are large enough that stochastic fluctuations in the $n_i$ (``drift'') can be ignored. In particular, we do not evaluate the initial stochastic behaviour of mutant lineages while they are at low abundance. We derive deterministic equations for the expected change in the $n_i$ over time, leaving the evaluation of drift for future work.  

%When considering new mutations, we therefore restrict our attention to the earliest (lowest $n_i$) deterministic behavior of mutant lineages (the transition to deterministic growth occurs at an abundance $n_i$ of order equal to their inverse expected absolute growth rate; \citealt{uecker_2011}).

\begin{figure}
\centering
\includegraphics[scale=0.8]{lottery.pdf}
\caption{\label{fig:lottery} Each iteration of our model has three elements. First, propagules are produced by adults and dispersed at random (only propagules landing on unoccupied territories are shown). Some territories may receive zero propagules. Lottery competition then occurs in each unoccupied territory that receives a propagule (only illustrated in one territory). Each type has a probability proportional to $c_i x_i$ of securing a given territory, where $c_i$ measures competitive ability and $x_i$ is the number of propagules that disperse there. In the illustrated territory, the black type disperses more propagules but is a poorer competitor. Territories are then made available by deaths among those adults present at the start of the iteration (red crosses).}
\end{figure}

Each iteration, adults produce new offspring (``propagules''), $m_i$ of which disperse to unoccupied territories. We assume that adults cannot be ousted from their territories, so that $m_i$ only includes propagules landing on unoccupied territories. Propagules disperse at random over the unoccupied territories, regardless of distance from their parents, and independently of each other. There is no interaction between propagules (e.g. avoidance of territories crowded with propagules). Loss of propagules during dispersal is subsumed into $m_i$. We assume that each adult produces a constant number $b_i$ of successfully dispersing propagules; the loss of propagules due to dispersal to occupied territories then implies $m_i=b_i(1-N/T)n_i$. Note that due to our assumption of uniform dispersal, the parameter $b_i$ can be thought of as a measure of ``colonization ability'', which combines fecundity and dispersal ability \citep{levins_71,tilman_94,bolker_99}. 

The number of individuals of the $i$'th type landing in any particular territory is denoted $x_i$. We assume that $x_i$ follows a Poisson distribution $p_i(x_i)=l_i^{x_i} e^{-l_i}/x_i!$, where $l_i=m_i/U$ is the mean territorial propagule density. This is approximation becomes exact when the $n_i$ are large enough that drift in $n_i$ can be ignored (Appendix A).

When multiple propagules land on the same territory, the victor is determined by lottery competition: type $i$ wins a territory with probability $c_i x_i/\sum_j c_j x_j$, where $c_i$ is a constant representing relative competitive ability (Fig. \ref{fig:lottery}). We expect that a fraction $p_1(x_1)\ldots p_G(x_G)$ of the $U$ unoccupied territories will have the propagule composition $x_1,\ldots,x_G$. type $i$ is expected to win $c_i x_i/\sum_j c_j x_j$ of these. Ignoring fluctuations about these two expectations (due to our no-drift, large $T$, large $n_i$ approximation), type $i$'s territorial acquisition is given by
\begin{equation}
\Delta_+ n_i(t)=U(t)\sum_{x_1,\ldots,x_G} \frac{c_i x_i}{\sum_j c_j x_j} p_1(x_1)\ldots p_G(x_G), \label{eq:growthsumuncoupled}
\end{equation}
in our extended lottery model, where the sum only includes territories with at least one propagule present.

Finally, we assume that mortality only occurs in adults (Fig.~\ref{fig:lottery}; setting aside the juvenile deaths implicit in territorial contest), and at a constant, type-specific per-capita rate $0\leq d_i\leq 1$, so that the overall change in type abundances is
\begin{equation}
\Delta n_i(t)=\Delta_+ n_i(t)-d_i n_i(t). \label{eq:delttot}
\end{equation}

\subsection*{Connection to the Wright-Fisher and classic lottery models}

In the classic lottery model \citep{chesson_1981}, unoccupied territories are assumed to be saturated with propagules from every type $l_i\gg 1$. From the law of large numbers, the composition of propagules in each territory will then not deviate appreciably from the mean composition $l_1,l_2,\ldots,l_G$ ($G$ is the number of types present), and so the probability that type $i$ wins any particular unoccupied territory is approximately $c_i l_i/\sum_j c_j l_j$. Then the numbers of territories won by each type $\Delta_+ n_1,\Delta_+ n_2,\ldots,\Delta_+ n_G$ follow a multinomial distribution with $U$ trials and success probabilities $\frac{c_1 l_1}{\sum_j c_j l_j},\frac{c_2 l_2}{\sum_j c_j l_j},\ldots,\frac{c_G l_G}{\sum_j c_j l_j}$, respectively. Type $i$ is expected to win $c_i l_i/\sum_j c_j l_j$ of the $U$ available territories, and deviations from this expected outcome are small (since $T$ is large by assumption), giving 
\begin{equation}
\Delta_+ n_i(t)=\frac{c_i l_i}{\sum_j c_j l_j}U(t)=\frac{c_i l_i}{\overline{c}L}U(t), \label{eq:lottery}
\end{equation}
where $\overline{c}=\sum_j c_j m_j/M$ is the mean propagule competitive ability for a randomly selected propagule, $L=M/U$ is the total propagule density and $M=\sum_j m_j$ is the total number of propagules. 

Eq. \eqref{eq:lottery} breaks down for types with low propagule density ($l_i\ll 1$) because territorial acquisition is then not correctly represented by a lottery in each territory with the mean propagule density. Instead, a rare type's propagules only make it to a few territories where at least one of its propagule present. In our extension of the classic lottery model, we correct (Eq.~\ref{eq:delttot}) to account for this.
 
There is a close connection between the classic lottery model and the Wright-Fisher model of genetic drift \citep{svardal_2015}. In the Wright-Fisher model, type abundances are sampled each generation from a multinomial distribution with success probabilities $w_i n_i/\sum_j w_j n_j$, where $w$ is relative fitness and the $n_i$ are  type abundances in the preceding generation. Population size $N$ remains constant. This is equivalent to the classic lottery model with non-overlapping generations ($d_i=1$ for all $i$) and relative fitness given by $w_i=b_i c_i$ i.e. a product of fecundity and viability \citep[pp. 185]{crow_1970}. Thus, the classic lottery model is essentially the Wright-Fisher model extended to allow overlapping generations, but ignoring drift. This means that our extension of the classic lottery model to arbitrary densities represents a density-dependent generalization of the Wright-Fisher model.

\section*{Results}

\subsection*{Mean-field approximation of the density-dependent lottery}

Eq. \eqref{eq:growthsumuncoupled} involves an expectation over the time-dependent dispersal distributions $p_i$, and is thus too complicated to give intuition about the dynamics of density-dependent lottery competition. We now evaluate this expectation using a ``mean field'' approximation. 

Similarly to the high-$l_i$ approximation of classic lottery model, we replace the $x_i$ with appropriate mean values, although we cannot simply replace $x_i$ with $l_i$. For a type with low propagule density $l_i\ll 1$, we have $x_i=1$ in the territories where its propagules land, and so its growth comes entirely from territories which deviate appreciably from $l_i$. To account for this, we separate Eq. \eqref{eq:growthsumuncoupled} into $x_i=1$ and $x_i>1$ parts. Our more general mean field approximation only requires that there are no large discrepancies in competitive ability (i.e. we do not have $c_i/c_j\gg 1$ for any two types). We obtain (details in Appendix B)
\begin{equation}
\Delta_+ n_i(t)\approx \left[e^{-L}+(R_i+A_i)\frac{c_i}{\overline{c}}\right]l_i U(t), \label{eq:master}
\end{equation}
where
\begin{equation}
R_i=\frac{\overline{c}e^{-l_i}(1-e^{-(L-l_i)})}{c_i +\frac{\overline{c}L- c_il_i}{L-l_i}\frac{L-1+e^{-L}}{1-(1+L)e^{-L}}},\nonumber \label{eq:Dr}
\end{equation}
and
\begin{equation}
A_i=\frac{\overline{c}(1-e^{-l_i})}{\frac{1-e^{-l_i}}{1-(1+l_i)e^{-l_i}}c_il_i+\frac{\overline{c}L- c_il_i}{L-l_i}\left(L\frac{1-e^{-L}}{1-(1+L)e^{-L}}-l_i\frac{1-e^{-l_i}}{1-(1+l_i)e^{-l_i}}\right)}. \nonumber \label{eq:Da}
\end{equation}

Comparing Eq. \eqref{eq:master} to Eq. \eqref{eq:lottery}, the classic lottery per-propagule success rate $c_i/\overline{c}L$ has been replaced by three separate terms. The first, $e^{-L}$, accounts for propagules which land alone on unoccupied territories; these territories are won without contest. The second, $R_i c_i/\overline{c}$ represents competitive victories when the $i$ type is a rare invader in a high density population, determining its invasion fitness \citep{metz_1992}. The third term, $A_i c_i/\overline{c}$, represents competitive victories when the $i$ type is abundant. The relative importance of these three terms varies with both the overall propagule density $L$ and the relative propagule frequencies $m_i/M$. If $l_i\gg 1$ for all types, we recover the classic lottery model (only the $A_ic_i/\overline{c}$ term remains, and $A_i\rightarrow 1/L$). Note that not all unoccupied territories are claimed each iteration, since under Poisson dispersal a fraction $e^{-L}$ remain unoccupied; total population density thus obeys 
\begin{equation}
\Delta N=(1-e^{-L})U-\sum_i d_i n_i \label{eq:Nmaster}
\end{equation}

Fig.~\ref{fig:simcomp} shows that Eq. \eqref{eq:master} and its components closely approximate simulations of the density-dependent lottery model over a wide range of propagule densities. Two types are present, one of which is at low frequency. The growth of the low-frequency type relies crucially on the low-density competition term $R_i c_i/\overline{c}$. On the other hand, $R_i c_i/\overline{c}$ is negligible for the high-frequency type, which depends instead on high density territorial victories. Fig.~\ref{fig:simcomp} also shows the breakdown of the classic lottery model at low propagule densities.

\begin{figure}
\centering
\includegraphics[scale=0.8]{simulationcomparison.pdf}
\caption{\label{fig:simcomp} Comparison of mean field approximation Eq. \eqref{eq:master} with simulations. Per-propagule success probability $\Delta_+ n_i/l_i U$ from the classic lottery model, individual-based simulations of random dispersal and lottery competition, and Eq. \eqref{eq:master} and its three components. Two types are present, a rare type with $c_1=1.5$, and a common type with $c_2=1$. Simulation points are almost invisible in for the common type due to near exact agreement with Eq. \eqref{eq:master}. Dashed lines in show the breakdown of the classic lottery model. Parameters: $m_1=10^4$ and $m_2=9\times 10^4$ and $U$ varies between $5\times 10^3$ and $10^6$.} 
\end{figure}

\subsection*{$K$-selection versus relative fitness}

We now compare the density-dependent lottery model from the previous section with MacArthur's analysis of selection in crowded environments \citep{macarthur_1967}. MacArthur considers a population with two types that have densities $n_1$ and $n_2$ subject to density-dependent growth described by
\begin{equation}
\frac{d n_1}{d t}=f_1(n_1,n_2)\qquad\frac{d n_2}{d t}=f_2(n_1,n_2). \label{eq:macgeneral}
\end{equation}
The environment is assumed to remain constant apart from the type densities. The functions $f_1$ and $f_2$ must decline to zero if $n_1$ or $n_2$ are sufficiently large, because no population has unlimited resources. This defines the nullclines $f_1(n_1,n_2)=0$ and $f_2(n_1,n_2)=0$ in $(n_1,n_2)$ space. The outcome of selection is then determined by the relationship between these nullclines. Specifically, a type will be excluded if its nullcline is completely contained in the region bounded by the other type's nullcline. In other words, for a type to have the possibility of persisting, it must be able to keep growing to higher densities than the other type can tolerate in some region of $(n_1,n_2)$ space (Fig.~\ref{fig:Ksel}a).

To formalize the relationship between nullclines, MacArthur used the symbol ``$K$'' to label the four intersection points of the nullclines with the $n_1$ and $n_2$ axes, specifically $f_1(K_{11},0)=0$, $f_1(0,K_{12})=0$, $f_2(0,K_{22})=0$ and $f_2(K_{21},0)=0$. These $K$ values determine whether a region of higher-density growth exists for each type, provided that the nullclines are close to being straight lines. Note that only $K_{11}$ and $K_{22}$ are saturation densities akin to the $K$ parameter in the logistic model; following widespread convention, we will refer to selection on these saturation densities as ``$K$-selection'' (Fig.~\ref{fig:Ksel}a). The other intersection points, $K_{12}$ and $K_{21}$, are related to competition between types. For instance, in the Lotka-Volterra competition model we have
\begin{align}
f_1(n_1,n_2) = r_1(1-\alpha_{11}n_1-\alpha_{12}n_2)n_1\nonumber\\
f_2(n_1,n_2) = r_2(1-\alpha_{22}n_1-\alpha_{21}n_2)n_2\label{eq:LV}
\end{align}
where $\alpha_{11}=1/K_{11}$ and $\alpha_{22}=1/K_{22}$ measure competitive effects within each type, while $\alpha_{12}=1/K_{12}$ and $\alpha_{21}=1/K_{21}$ measure competitive effects on the first type due to the second (Fig.~\ref{fig:LVvslottery}a). 

Thus, when MacArthur concludes that  ``fitness is $K$'' in crowded populations \citep[pp. 149]{macarthur_1967}, the meaning is that selection either favors the ability to keep growing at ever higher densities (moving a type's own nullcline outwards), or the ability to suppress the growth of competitors at lower densities (moving the nullcline of competitors inwards) \citep{gill_1974}. This general idea applies even if the nullclines are nonlinear to such an extent that the ``$K$'' values themselves do not give much information about the regions of high-density growth.

\begin{figure}
\centering
\includegraphics[scale=0.8]{LVvslottery.pdf}
\caption{\label{fig:LVvslottery} Selection between types with identical saturation density but different inter-type competitive ability. (a) Lotka-Volterra competition (Eq.~\ref{eq:LV}) with $r_1=r_2=1$, $\alpha_{11}=\alpha_{22}=1$, $\alpha_{12}=0.9$ and $\alpha_{21}=1.2$. Trajectories do not follow the line $N=K_{11}=K_{22}$. (b) Lottery competition (Eq.~\ref{eq:master}) with $b_1=b_2=5$, $d_1=d_2=0.1$ and $c_1/c_2=5$. Trajectories converge on the line $N=K_{11}=K_{22}$.}
\end{figure}

It is obvious from Eq.~\eqref{eq:LV} that selection can favor a superior competitor in a crowded population even if its saturation density is the same as, or lower than that of the other types present. However, note that the Lotka-Volterra model still closely couples selection to population density. Fig.~\ref{fig:LVvslottery}a shows Lotka-Volterra selection between two types with the same saturation density ($\alpha_{11}>\alpha_{22}$, $\alpha_{21}>\alpha_{12}$). Even though the initial and final densities of a sweep are the same, density is not constant during a sweep. Only a highly restricted subset of $r$ and $\alpha$ values will keep $N$ constant over a selective sweep (further details in Appendix C). Intuitively, for one type to exclude another with the same saturation density, inter-type competitive effects must be stronger than intra-type competitive effects, causing a dip in $N$ over the sweep. 

By contrast, if one type in our density-dependent lottery model has a $c$ advantage but birth and death rates are identical, the density trajectories converge on the line of constant density equal to the saturation density Fig.~\ref{fig:LVvslottery}b. Selection then occurs purely along this line, uncoupled from the density regulation of $N$. In other words, once the population reaches demographic equilibrium, it behaves indistinguishably from a constant-$N$ relative fitness model. More generally, the competitive ability trait $c$ does not directly affect population density (this can be seen formally in Eq.~\eqref{eq:Nmaster}), since $c$ only affects which type wins a territory, not whether a territory is won at all. This is all perfectly consistent with MacArthur's general argument. 

\section*{Type-specific responses to density and the strength of selection}

We are now in a position to analyze the validity of relative fitness models more explicitly. In the previous section we showed that selection and the regulation of population density can be completely independent of each other even if population growth is density-regulated, and moreover that MacArthur's argument for the $r$/$K$ scheme never precluded this possibility. Nevertheless, selection and density regulation \textit{are} intimately connected in widely used models of population growth. To understand why this poses a problem for Eq.~\eqref{eq:canonical}, consider the simple birth-death model \cite[pp. 20]{kostitzin_1939} 
\begin{equation}
\frac{d n_i}{dt}=(b_i -\delta_iN) n_i \label{eq:simplebirthdeath}
\end{equation}
where $\delta_i$ is the per-capita increase in type $i$'s mortality rate due to crowding (for simplicity, there are no deaths when uncrowded). Then, starting from a monomorphic population, the frequency of an adaptive $\delta$-variant $\delta_i\rightarrow \delta_i(1-\epsilon)$ obeys 
\begin{equation}
\frac{d p_i}{dt}=\epsilon \delta_i N p_i(1-p_i). \label{eq:Ndependentsweep}
\end{equation}
The selection coefficient $s=\epsilon \delta_i N$ thus depends on density (compare \cite[pp. 29]{crow_1970}). On the other hand, the frequency of an adaptive $b$-variant $b_i\rightarrow b_i(1+\epsilon)$ will exactly obey Eq.~\eqref{eq:canonical} with $s=\epsilon b_i$, independent of density.

In practice the density dependence in Eq.~\eqref{eq:Ndependentsweep} only matters if $N$ changes substantially during a sweep. This can easily occur if a population is far from demographic equilibrium (we return to this scenario in the Discussion). A more serious objection to Eq.~\eqref{eq:canonical} is that it is internally inconsistent even in crowded populations where $N$ has reached equilibrium, because adaptive sweeps in density-dependent traits then induce a change in $N$ and also change themselves in response to $N$. To quantify how serious this objection is, we need to account for how much $N$ changes over a sweep. In Eq.~\eqref{eq:simplebirthdeath}, the saturation density is $N=b_i/\delta_i$, and so the selection coefficient increases from $s_{\rm initial}$ to $s_{\rm final}=s_{\rm initial}/(1-\epsilon)$ over a sweep (Fig.~\ref{fig:strengthofselection}). Thus, Eq.~\eqref{eq:canonical} breaks down if selection is sufficiently strong on the density-dependent mortality rate, with proportional effects of $\epsilon>0.2$ inducing substantial deviations from constant selection. 

\begin{figure}
\centering
\includegraphics[scale=0.8]{strengthofselection.pdf}
\caption{\label{fig:strengthofselection} Proportional change in the selection coefficient over a ``$K$-selection''-type sweep for a type that experiences proportionally $1-\epsilon$ fewer deaths induced by crowding. The population is in demographic equilibrium at the start of the sweep.}
\end{figure}

Let us now contrast the simple linearly density-dependent model Eq.~\eqref{eq:Ndependentsweep} with our density-dependent lottery. As we have seen in our lottery model, the importance of $c$ depends on density, yet since $c$ has no effect on density, $s$ is exactly constant for $c$ sweeps. For $b$ and $d$ sweeps, $s$ is also exactly consant, because the overall regulation of population density applies equally to all types \citep{prout_1980}. To see this formally, we set $c_i$ to be the same for all types in Eq.~\eqref{eq:master} which yields
\begin{equation}
\Delta n_i = \frac{m_i}{M}(1-e^{-L})U-d_i n_i = \left(b_i\frac{1-e^{-L}}{L}-d_i \right) n_i.
\end{equation}
Thus, the density-regulation of population size occurs via the juvenile recruitment fraction $(1-e^{L})/L$. This applies uniformly to all types and therefore does not induce density-dependence in selective advantages related to $b$ or $d$. Note that our lottery model accounts for two key aspects of the interaction between selection and demography: selection is density-dependent, and density will change as a result of selection on absolute fitness traits. And yet pure $b$, $c$ and $d$ sweeps all obey Eq.~\eqref{eq:canonical} exactly. 

The reason that our lottery model behaves so differently from Eq.~\eqref{eq:simplebirthdeath} is that each type can respond differently to density via $c$ without affecting the regulation of population density. That is, our lottery model separates density-dependent selection from ``selection-dependent density'' \citep{prout_1980}, two distinct effects that are conflated in simple growth models with linearly-density-dependent per-capita growth rates. If types in the lottery model differ in $c$ as well as $b$ or $d$, then the lottery model will exhibit similar complications as Fig.~\ref{fig:strengthofselection}.

\section*{Discussion}

%This  also seems to be why the density-dependent selection literature expended such considerable effort to showing that evolution under density-dependent selection optimizes population size in some sense \citep{roughgarden_1979} 


It is important to recognize that MacArthur's argument does not imply a dichotomy between $r$-selection (uncrowded) and $K$-selection (crowded), with the latter taken to mean selection for greater saturation density. A better dichotomy is between interaction-dependent selection and interaction-independent selection. That is, selective shifts in frequency are a result of differences in absolute growth rates, but these differences can arise in two logically distinct ways: 1) some types expand more rapidly in the absence of interactions between individuals or 2) some types are superior in their interactions with other types. Population density is a key factor controlling whether individuals interact, thereby setting the relative contributions of these forms of selection. 

%In the case of exploitation competition for consumable resources, intra- and inter-type competition are connected via resource use ``efficiency'', but obviously this need not be true more generally \citep{gill_1974,case_1974}.

%To avoid confusion with other forms of ''$K$-selection'', selection for competitive ability has been called ``$\alpha$-selection'' after the competition coefficients in the Lotka-Volterra equation \citep{gill_1974,case_1974,joshi_2001}. However, competitive success as measured by $\alpha$ (i.e. the per-capita effect of one type on another type's growth rate) is only partly determined by individual competitive ability --- in the presence of age-structured competition and territoriality, it also includes the ability of each type to produce contestants i.e. $b$ in our model. 

One potential limitation of our model as a general-purpose model of density-dependent selection is its restriction to interference competition between juveniles for durable resources (lottery recruitment to adulthood), analogous to the ubiquitous assumption of viability selection in population genetics \citep[p. 45]{ewens_2004}. In some respects this is the complement of consumable resource competition models, which restrict their attention to indirect exploitation competition, typically without age structure \citep{tilman_1982}. In the particular case that consumable resources are spatially localized (e.g. due to restricted movement through soils), resource competition and territorial acquisition effectively coincide, and in principle resource competition could be represented by a competitive ability $c$ (or conversely, $c$ should be derivable from resource competition). The situation is more complicated if the resources are well-mixed, since, in general, resource levels then need to be explicitly tracked. It seems plausible that explicit resource tracking may not be necessary when the focus is on the evolution of similar types that use identical resources rather than the stable co-existence of widely differing species with different resource preferences \citep{ram_2016}. We are not aware of any attempts to delineate conditions under which explicit resource tracking is unnecessary even if it is assumed that community structure is ultimately determined by competition for consumable resources. More work is needed connecting resource competition models to the density-dependent selection literature, since most of the former has to date been focused on narrower issues of the role of competition at low resource availability and in the absence of direct interactions between organisms at the same trophic level \citep{aerts_1999,davis_1998,tilman_2007}.  

While our model can be applied to species rather than types (e.g. ecological invasions), our focus is type evolution i.e. the change in allele frequencies over time. Our assumption that there are no large $c$ discrepancies (section ``Mean field approximation'') amounts to a restriction on the amount of genetic variation in $c$ in the population. Since beneficial mutation effect sizes will typically not be much larger than a few percent, large $c$ discrepancies can only arise if the mutation rate is extremely large, and so the assumption will not be violated in most cases. However, this restriction could become important when looking at species interactions rather than type evolution.

Another issue with the constant-$N$ relative fitness description of selection is that it precludes consideration of longer-term aspects of the interplay between evolution and ecology such as population extinction. Adaptive dynamics currently provides a powerful framework for adresing the complex feedbacks between evolutionary change and population density. However, the focus of adatpive dynamics is trait evolution rather than the underlying genetics, and in particular, selective sweeps are typically subsumed into effectively-instantaneous ``trait substitutions''. We emphasize that our focus here has been the description of selection itself, which is particularly critical for making sense of evolution at the genetic level.
 

%\section*{Acknowledgments}

%We thank Peter Chesson and Joachim Hermisson for many constructive comments on this manuscript. This work was financially supported by the National Science Foundation (DEB-1348262).

\bibliographystyle{plainnat}
\bibliography{reference} 

\section*{Appendix A: Poisson approximation}

For simplicity of presentation, we have assumed a Poisson distribution for the $x_i$ as our model of dispersal. Strictly speaking, the total number of $i$ propagules $\sum x_i$ (summed over unoccupied territories) is then no longer a constant $m_i$, but fluctuates between generations for a given mean $m_i$, which is more biologically realistic. Nevertheless, since we do not consider the random fluctuations in type abundances here, and for ease of comparison with the classic lottery model, we ignore the fluctuations in $m_i$. Instead we focus, on Poisson fluctuations in propagule composition in each territory. 

In the exact model of random dispersal, the counts of a type's propagules across unnocupied territories follows a multinomial distribution with dimension $U$, total number of trials equal to $m_i$, and equal probabilities $1/U$ for a propagule to land in a given territory. Thus, the $x_i$ in different territories are not independent random variables. However, for sufficiently large $U$ and $m_i$, this multinomial distribution for the $x_i$ across territories is closely approximated by a product of independent Poisson distributions for each territory, each with rate parameter $l_i$ \citep[Theorem 1]{arenbaev_1977}. Since we are ignoring finite population size effects, we effectively have $T\rightarrow \infty$, in which case $U$ can be only be small enough to violate the Poisson approximation if there is vanishing population turnover, and then the dispersal distribution is irrelevant anyway. Likewise, in ignoring stochastic finite population size for the $n_i$, we have effectively already assumed that $m_i$ is large enough to justify the Poisson approximation (the error scales as $1/\sqrt{m_i}$; \citealt{arenbaev_1977}).

\section*{Appendix B: Derivation of growth equation}

We separate the right hand side of Eq.~\eqref{eq:growthsumuncoupled} into three components $\Delta_+ n_i = \Delta_u n_i+\Delta_r n_i+\Delta_a n_i$ which vary in relative magnitude depending on the propagule densities $l_i$. Following the notation in the main text, the Poisson distributions for the $x_i$ (or some subset of the $x_i$) will be denoted $p$, and we use $P$ as a general shorthand for the probability of particular outcomes.

\subsection*{Growth without competition}

The first component, $\Delta_u n_i$, accounts for territories where only one focal propagule is present $x_i=1$ and $x_j=0$ for $j\neq i$ ($u$ stands for ``uncontested''). The proportion of territories where this occurs is $l_i e^{-L}$, and so 
\begin{equation}
\Delta_u n_i=Ul_i e^{-L}=m_i e^{-L}.
\end{equation}

\subsection*{Competition when rare}

The second component, $\Delta_r n_i$, accounts for territories where a single focal propagule is present along with at least one non-focal propagule ($r$ stands for ``rare'') i.e. $x_i=1$ and $X_i\geq 1$ where $X_i=\sum_{j\neq i} x_j$ is the number of nonfocal propagules. The number of territories where this occurs is $Up_i(1)P(X_i\geq 1)=b_i n_i e^{-l_i}(1-e^{-(L-l_i)})$. Thus 
\begin{equation}
\Delta_r n_i = m_i e^{-l_i}(1-e^{-(L-l_i)})\left\langle  \frac{c_i}{c_i +\sum_{j\neq i} c_j x_j } \right\rangle_{\tilde{p}},  \label{eq:deltr}
\end{equation}
where $\langle \rangle_{\tilde{p}}$ denotes the expectation with respect to $\tilde{p}$, and $\tilde{p}$ is the probability distribution of nonfocal propagule abundances $x_j$ \textit{after} dispersal, in those territories where exactly one focal propagule, and at least one non-focal propagule, landed. 

Our ``mean field'' approximation is to replace $x_j$ with its mean in the last term in Eq.~\eqref{eq:deltr},
\begin{equation}
\left\langle\frac{c_i}{c_i +\sum_{j\neq i} c_j x_j}\right\rangle_{\tilde{p}}\approx \frac{c_i}{c_i +\sum_{j\neq i} c_j \langle x_j\rangle_{\tilde{p}}}.\label{eq:meanfieldr}
\end{equation}
Below we justify this replacement by arguing that the standard deviation $\sigma_{\tilde{p}}(\sum_{j\neq i} c_j x_j)$ (with respect to $\tilde{p}$), is much smaller than $\langle\sum_{j\neq i} c_j x_j\rangle_{\tilde{p}}$.

We first calculate $\langle x_j \rangle_{\tilde{p}}$. Let $X=\sum_j x_j$ denote the total number of propagules in a territory and ${\mathbf x_i}=(x_1,\ldots,x_{i-1},x_{i+1}\ldots,x_G)$ denote the vector of non-focal abundances, so that $p({\mathbf x_i})=p_1(x_1)\ldots p_{i-1}(x_{i-1})p_{i+1}(x_{i+1})\ldots p_G(x_G)$. Then, $\tilde{p}$ can be written as
\begin{align}
\tilde{p}({\mathbf x_i})&=p({\mathbf x_i}|X\geq 2,x_i=1)\nonumber\\
&=\frac{P({\mathbf x_i},X\geq 2|x_i=1)}{P(X\geq 2)}\nonumber\\
&=\frac{1}{1-(1+L)e^{-L}}\sum_{X=2}^{\infty} P(X) p({\mathbf x_i}|X_i=X-1),
\end{align}
and so
\begin{align}
\langle x_j \rangle_{\tilde{p}}&=\sum_{\mathbf x_i} \tilde{p}({\mathbf x_i})x_j\nonumber\\
&=\frac{1}{1-(1+L)e^{-L}}\sum_{X=2}^{\infty} P(X) \sum_{\mathbf x_i} p({\mathbf x_i}|X_i=X-1)x_j.
\label{eq:raremonster1}
\end{align}
The inner sum over ${\mathbf x_i}$ is the mean number of propagules of a given nonfocal type $j$ that will be found in a territory which received $X-1$ nonfocal propagules in total, which is equal to $\frac{l_j}{L-l_i}(X-1)$. Thus, 
\begin{align}
\langle x_j \rangle_{\tilde{p}}&=\frac{l_j}{1-(1+L)e^{-L}}\frac{1}{L-l_i}\sum_{k=2}^{\infty} P(X) (X-1)\nonumber\\
&=\frac{l_j}{1-(1+L)e^{-L}}\frac{L-1+e^{-L}}{L-l_i},
\label{eq:meanxjrare}
\end{align}
where the last line follows from $\sum_{X=2}^{\infty} P(X)(X-1)=\sum_{X=1}^{\infty} P(X)(X-1)=\sum_{X=1}^{\infty} P(X)X-\sum_{X=1}^{\infty}P(X)$.

The exact analysis of the fluctuations in $\sum_{j\neq i} c_j x_j$ is complicated because the $x_j$ are not independent with respect to $\tilde{p}$. These fluctuations are part of the ``drift'' in type abundances which we leave for future work. Here we use the following approximation to give some insight into the magnitude of these fluctuations and also the nature of the correlations between the $x_j$. We replace $\tilde{p}$ with $\tilde{q}$, defined as the ${\mathbf x_i}$ Poisson dispersal probabilities conditional on $X_i\geq1$ (which are independent). The distinction between $\tilde{p}$ with $\tilde{q}$ will be discussed further below. The $\tilde{q}$ approximation gives $\langle x_j \rangle_{\tilde{q}}=\langle x_j \rangle_p/C=l_j/C$, 
\begin{align}
\sigma_{\tilde{q}}^2(x_j)&=\langle x_j^2 \rangle_{\tilde{q}}-\langle x_j \rangle_{\tilde{q}}^2\nonumber\\
&=\frac{1}{C}\langle x_j^2 \rangle_p-\frac{l_j^2}{C^2}\nonumber \\
&=\frac{1}{C}(l_j^2 + l_j)-\frac{l_j^2}{C^2}\nonumber \\
&=\frac{l_j^2}{C}\left(1-\frac{1}{C}\right)+\frac{l_j}{C},\label{eq:varr}
\end{align}
and 
\begin{align}
\sigma_{\tilde{q}}(x_j,x_k)&=\langle x_j x_k \rangle_{\tilde{q}}-\langle x_j \rangle_{\tilde{q}}\langle x_k \rangle_{\tilde{q}}\nonumber\\
&=\frac{1}{C}\langle x_j x_k \rangle_p-\frac{l_jl_k}{C^2}\nonumber\\
&=\frac{l_j l_k}{C}\left(1-\frac{1}{C}\right),\label{eq:covr}
\end{align}
where $C=1-e^{-(L-l_i)}$ and $j\neq k$. 

The exact distribution $\tilde{p}$ assumes that exactly one of the  propagules present in a given site after dispersal belongs to the focal type, whereas $\tilde{q}$ assumes that there is a focal propagule present before non-focal dispersal commences. As a result, $\tilde{q}$ predicts that the mean propagule density is greater than $L$ (in sites with only one focal propagule is present) when the focal type is rare and the propagule density is high. This is erroneous, because the mean number of propagules in every site is $L$ by definition. Specifically, if $L-l_i \approx L\gg 1$, then the mean propagule density predicted by $\tilde{q}$ is approximately $L+1$. The discrepancy causes rare invaders to have an intrinsic rarity disadvantage (territorial contests under $\tilde{q}$ are more intense than they should be). In contrast, Eq. \eqref{eq:meanxjrare} correctly predicts that there are on average $\sum_{j\neq i}\langle x_j \rangle_{\tilde{p}}\approx L-1$ nonfocal propagules because $\tilde{p}$ accounts for potentially large negative covariances between the $x_j$ ``after dispersal''. By neglecting the latter covariences, $\tilde{q}$ overestimates the fluctuations in $\sum_{j\neq i} c_j x_j$; thus $\tilde{q}$ gives an upper bound on the fluctuations. The discrepancy between $\tilde{q}$ and $\tilde{p}$ will be largest when $L$ is of order $1$ or smaller, because then the propagule assumed to already be present under $\tilde{q}$ is comparable to, or greater than, the entire propgaule density. 

Decomposing the variance in $\sum_{j\neq i} c_j x_j$,
\begin{equation}
\sigma_{\tilde{q}}^2(\sum_{j\neq i} c_j x_j)=\sum_{j\neq i}\left[c_j^2\sigma_{\tilde{q}}^2(x_j)+2\sum_{k>j, k\neq i}c_j c_k\sigma_{\tilde{q}}(x_j,x_k)\right],\label{eq:vartotr}
\end{equation}
and using the fact that $\sigma_{\tilde{q}}(x_j,x_k)$ and the first term in Eq. \eqref{eq:varr} are negative because $C<1$, we obtain an upper bound on the relative fluctuations in $\sum_{j\neq i} c_j x_j$, 
\begin{equation}
\frac{\sigma(\sum_{j\neq i} c_j x_j)}{\langle\sum_{j\neq i} c_j x_j\rangle}=C^{1/2}\frac{\left(\sum_{j\neq i}c_j^2 l_j+(1-1/C)\left(\sum_{j\neq i}c_j l_j\right)^2 \right)^{1/2}}{\sum_{j\neq i}c_j l_j}<C^{1/2}\frac{\left(\sum_{j\neq i}c_j^2 l_j\right)^{1/2}}{\sum_{j\neq i}c_j l_j}. \label{eq:cvr}
\end{equation}

Suppose that the $c_j$ are all of similar magnitude (their ratios are of order one). Then Eq.~\eqref{eq:cvr} is $\ll 1$ for the case when $L-l_i \ll 1$ (due to the factor of $C^{1/2}$), and also for the case when at least some of the nonfocal propagule densities are large $l_j\gg 1$ (since it is then of order $1/\sqrt{L-l_i}$). The worst case scenario occurs when $L-l_i$ is of order one. Then Eq.~\eqref{eq:cvr} gives a relative error of approximately $50\%$, which from our earlier discussion we know to be a substantial overestimate when $L$ is of order $1$. Our numerical results (Fig. \ref{fig:simcomp}) confirm that the relative errors are indeed small.

However, the relative fluctuations in $\sum_{j\neq i} c_j x_j$ can be large if some of the $c_j$ are much larger than the others. Specifically, in the presence of a rare, extremely strong competitor ($c_j l_j\gg c_{j'} l_{j'}$ for all other nonfocal types $j'$, and $l_j\ll 1$), then the RHS of Eq. \eqref{eq:cvr} can be large and we cannot make the replacement Eq.~\eqref{eq:meanfieldr}. 

Substituting Eqs. \eqref{eq:meanfieldr} and \eqref{eq:meanxjrare} into Eq.~\eqref{eq:deltr}, we obtain
\begin{equation}
\Delta_r n_i\approx m_i R_i\frac{c_i}{\overline{c}}, \label{eq:deltrfinal}
\end{equation}
where $R_i$ is defined in Eq.~\eqref{eq:Dr}.

\subsection*{Competition when abundant}

The final contribution, $\Delta_a n_i$, accounts for territories where two or more focal propagules are present ($a$ stands for ``abundant"). Similarly to Eq.~\eqref{eq:deltr}, we have 
\begin{equation}
\Delta_a n_i=U(1-(1+l_i)e^{l_i})\left\langle \frac{c_i x_i}{\sum_j c_j x_j} \right\rangle_{\hat{p}}\label{eq:delta}
\end{equation}
where $\hat{p}$ is the probability distribution of both focal and nonfocal propagaule abundances \textit{after} dispersal in those territories where at least two focal propagules landed. 

Again, we argue that the relative fluctuations in $\sum c_j x_j$ are much smaller than $1$ (with respect to $\hat{p}$), so that,
\begin{equation}
\left\langle \frac{c_i x_i}{\sum_j c_j x_j} \right\rangle_{\hat{p}}\approx  \frac{c_i \langle x_i \rangle_{\hat{p}}}{\sum_j c_j \langle x_j\rangle_{\hat{p}}}.\label{eq:meanfielda}
\end{equation}
Following a similar procedure as for $\Delta_r n_i$, where the vector of propagule abundances is denoted ${\mathbf x}$, the mean focal type abundance is, 
\begin{align}
\langle x_i \rangle_{\hat{p}}&=\sum_{\mathbf x} x_i p(\mathbf x|x_i\geq 2)\nonumber \\
&=\sum_{x_i} x_i p(x_i|x_i\geq 2) \nonumber\\
&=\frac{1}{1-(1+l_i)e^{-l_i}}\sum_{x_i\geq 2} p(x_i)x_i\nonumber\\
&=l_i\frac{1-e^{-l_i}}{1-(1+l_i)e^{-l_i}}.
\end{align}
For nonfocal types $j\neq i$, we have
\begin{align}
\langle x_j \rangle_{\hat{p}}&=\sum_{\mathbf x} x_j p(\mathbf x|x_i\geq 2)\nonumber \\
&=\sum_{X}P(X|x_i\geq 2)\sum_{\mathbf x} x_j p({\mathbf x}|x_i\geq 2,X)\nonumber\\
&=\sum_{X}P(X|x_i\geq 2)\sum_{x_i} p(x_i|x_i\geq 2,X) \sum_{\mathbf x_i} x_j p(\mathbf x_i|X_i=X-x_i)\nonumber\\
&=\sum_{X}P(X|x_i\geq 2)\sum_{x_i}p(x_i|x_i\geq 2,X) \frac{l_j(X-x_i)}{L-l_i} \nonumber\\
&=\frac{l_j}{L-l_i}\left[\sum_{X}P(X|x_i\geq 2)X - \sum_{x_i}p(x_i|x_i\geq 2) x_i \right]\nonumber\\
&=\frac{l_j}{L-l_i}\left( L\frac{1-e^{-L}}{1-(1+L)e^{-L}}- l_i\frac{1-e^{-l_i}}{1-(1+l_i)e^{-l_i}}\right). 
\end{align}

To calculate the relative fluctuations in $\sum_{j\neq i} c_j x_j$, we use a similar approximation as for $\Delta_r n_i$: $\hat{p}$ is approximated by $\hat{q}$, defined as the ${\mathbf x}$ dispersal probabilities in a territory conditional on $x_i>2$ (that is, treating the $x_j$ as indepenent). All covariances between nonfocal types are now zero, so that $\sigma_{\hat{q}}^2(\sum c_j x_j)=\sum c_j^2 \sigma_{\hat{q}}^2(x_j)$, where $\sigma_{\hat{q}}^2(x_j)=l_j$ for $j\neq i$, and  
\begin{equation}
\sigma_{\hat{q}}^2(x_i)=\frac{l_i}{D}\left(l_i+1-e^{-l_i}-\frac{l_i}{D}\left(1-e^{-l_i}\right)^2\right),
\end{equation}
where $D= 1-(1+l_i)e^{-l_i}$, and 
\begin{equation}
\frac{\sigma_{\hat{q}}(\sum c_j x_j)}{\langle\sum c_j x_j\rangle} = \frac{\left(\sum_{j\neq i} c_j^2 l_j + c_i^2 \sigma_{\hat{q}}^2(x_i)\right)^{1/2}}{\sum_{j\neq i} c_j l_j + c_i l_i (1-e^{-l_i})/D} \label{eq:cva}.
\end{equation}

Similarly to Eq.~\eqref{eq:cvr}, the RHS of Eq. \eqref{eq:cva} is $\ll 1$ for the case that $L \ll 1$ (due to a factor of $D^{1/2}$), and also for the case when at least some of the propagule densities (focal or nonfocal) are large --- provided that $c_i$ and the $c_j$ are all of similar magnitude. Again, the worst case scenario occurs when $l_i$ and $L-l_i$ are of order $1$, in which case Eq. \eqref{eq:cva} is around $35\%$, which is again where the $\hat{q}$ approximation produces the biggest overestimate of the fluctuations in ${\mathbf x}$. Similarly to Eq.~\eqref{eq:cvr}, the RHS of \eqref{eq:cva} will not be $\ll 1$ in the presence of a rare, extremely strong competitor.  

Combining Eqs. \eqref{eq:delta} and \eqref{eq:meanfielda}, we obtain
\begin{equation}
\Delta_a n_i=m_i A_i \frac{c_i}{\overline{c}},
\end{equation}
where $A_i$ is defined in Eq.~\eqref{eq:Da}.

\section*{Appendix C: Total density in the Lotka-Volterra competition model}

Here we show that under the Lotka-Volterra model of competition, total density $N$ does not in general remain constant over a selective sweep in a crowded population even if the types have the same saturation density.

We assume $\alpha_{11}=\alpha_{22}$ and $N=1/\alpha_{11}$ and check whether it is then possible for $\frac{dN}{dt}$ to be zero in the sweep ($n_1,n_2 \neq 0$). Substituting these conditions into Eq.~\eqref{eq:LV}, we obtain 
\begin{align}
\frac{d n_1}{dt} = -r_1(\alpha_{12}-\alpha_{11})n_1n_2 \nonumber\\
\frac{d n_2}{dt} = -r_2(\alpha_{21}-\alpha_{22})n_1n_2
\end{align}
Adding these together, $\frac{dN}{dt}$ can only be zero if 
\begin{equation}
r_1(\alpha_{12}-\alpha_{11})+r_2(\alpha_{21}-\alpha_{22})=0. \label{eq:constNcondition}
\end{equation}
To get some intuition for Eq.~\eqref{eq:constNcondition}, suppose that a mutant arises with improved competitive ability but identical intrinsic growth rate and saturation density ($r_1=r_2$ and $\alpha_{11}=\alpha_{22}$). This could represent a mutation to an interference competition trait, for example \citep{gill_1974}. Then, according the above condition, for $N$ to remain constant over the sweep, the mutant must find the wildtype more tolerable than itself by exactly the same amount that the wildtype finds the mutant less tolerable than itself. This condition, and Eq.~\eqref{eq:constNcondition} more generally, are so restrictive that we can conclude that selective sweeps in the Lotka-Volterra competition model will generally involve non-constant $N$. 


\end{document}
