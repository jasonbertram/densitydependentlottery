\documentclass[11pt]{article}
\usepackage[sc]{mathpazo}
\usepackage{amsmath,pifont}
\usepackage{fullpage}
\usepackage[authoryear,sectionbib,sort]{natbib}
\linespread{1.7}
\usepackage[utf8]{inputenc}
\usepackage{lineno}

\usepackage{graphicx} 
\usepackage{tabularx,setspace}

\title{A lottery model of density-dependent selection}
%\author{Jason Bertram $^{1,\ast}$ \\ 
%Joanna Masel $^{1}$}

\date{}

\begin{document}

\maketitle

%\noindent{}1. Department of Ecology and Evolutionary Biology, University of Arizona, Tucson, AZ 85721.

%\noindent{}$\ast$ Corresponding author; e-mail: jbertram@email.arizona.edu.


\bigskip

%\textit{Manuscript elements}: 

\bigskip

\textit{Keywords}: r/K selection, absolute fitness, eco-evo, competition-colonization trade-off, fluctuating selection, storage effect.

\bigskip


\linenumbers{}
\modulolinenumbers[1]

\newpage{}

\section*{Abstract}


This excludes fundamental ecological factors such as dynamic population size or density-dependence from the most genetically-realistic treatments of evolution, a problem that inspired MacArthur's influential but problematic $r$/$K$ theory. 

Following the spirit of $r$/$K$-selection as a general-purpose theory of density-dependent selection

new model of density-dependent selection by generalizing the fixed-density classic lottery model of territorial acquisition to accommodate arbitrary population densities. 

We show that, with density dependence, co-existence is possible in the lottery model in a stable environment. 

Inspired by natural \textit{Drosophila} populations, we consider co-existence under strong, seasonally-fluctuating selection coupled to large cycles in population density, and show that co-existence (stable polymorphism) is promoted via a combination of the classic storage effect and density-regulated population growth. 

\newpage{}


\section*{Introduction}

There are a variety of different measures of fitness. Some widely used examples in evolutionary ecology are expected lifetime reproductive ratio $R_0$, intrinsic growth rate $r$, saturation population density (often labeled ``$K$'') \citep{benton_2000}, and invasion fitness \citep{metz_1992}. In addition, ``relative fitness'' is the standard in much of evolutionary biology, particularly evolutionary genetics, where attention is generally restricted to relative genotypic proportions \cite[pp. 468]{barton_2007}. 

Despite this variety, in principle all fitness measures are ultimately ways to quantify an individual's potential for reproduction and survival \citep{metcalf_2007}. Yet the shifting-frequency relative view of fitness stands out as particularly divorced from population ecology. 

In uncrowded populations, there is no discrepancy. Relative fitness simply represents  differences in $r$ ($r$-selection) \citep[pp. 26]{crow_1970}.

Absolute measures of fitness like $r$ and $K$ are derived directly from population ecology. By contrast,  

This discrepancy runs deeper than a simple difference in emphasis on absolute densities rather than relative frequencies. 

 The discrepancy emerges in crowded populations. Then the relative fitness description of selection assumes that $N$ is fixed, and fitness strictly involves the ability to increase in frequency at the expense of others (e.g. in the Wright-Fisher model). On the other hand, from quite general population ecological considerations, MacArthur famously argued that ``fitness is $K$'' in crowded populations ($K$-selection) \citep[pp. 149]{macarthur_1967}. To appreciate the import of MacArthur's argument, and see the nature of the discrepancy between these views of selection, we now briefly review MacArthur's argument. 

MacArthur considers two types subject to density-dependent population growth described by
\begin{equation}
\frac{d n_1}{d t}=f_1(n_1,n_2)\qquad\frac{d n_2}{d t}=f_2(n_1,n_2), \label{eq:macgeneral}
\end{equation}
where $f_1$ and $f_2$ are strictly decreasing functions of the population densities $n_1$ and $n_2$. Plotting the nullclines $f_1(n_1,n_2)=0$ and $f_2(n_1,n_2)=0$, it can be seen that a type will be excluded if its nullcline is completely contained in the region bounded by the other type's nullcline. For a type to have the possibility of persisting, it must be able to grow in some region of $(n_1,n_2)$ space at higher total population density $N=n_1+n_2$ than the other type (Fig.~\ref{fig:Ksel}a). 

MacArthur defines four ``carrying capacities'' as the intersection points of the nullclines with the $n_1$ and $n_2$ axes (i.e. $f_1(K_{11},0)=0$, $f_1(0,K_{12})=0$, $f_2(0,K_{22})=0$ and $f_2(K_{21},0)=0$). Assuming that the nullclines are close to linear, these $K$ values determine whether a region of higher-density growth exists for each type. Note that only $K_{11}$ and $K_{22}$ are saturation densities analogous to ``$K$'' in the logistic model; $K_{12}$ and $K_{21}$ represent the effects of competition between types. For instance, under Lotka-Volterra competition we have $f_1(n_1,n_2)=r_1(1-\alpha_{11}n_1-\alpha_{12}n_2)n_1$. Then the nullclines are linear and $\alpha_{11}=1/K_{11}$ measures intra-type competitive effects, while $\alpha_{12}=1/K_{12}$ measures inter-type competitive effects.  Thus, to summarize, ``fitness is $K$'' in the sense that selection in crowded populations either favors the ability to keep growing at ever higher densities (moving a type's own nullcline outwards), or the ability to suppress the growth of competitors at lower densities (moving the nullcline of competitors inwards). This general idea applies even if the nullclines are nonlinear to such an extent that the ``$K$'' values themselves do not give much information about the regions of high-density growth.

\begin{figure}
\centering
\includegraphics[scale=0.8]{Kplot.pdf}
\caption{\label{fig:Ksel} (a) MacArthur's dynamical argument for why ``fitness is $K$'' in crowded environments, using a Lotka-Volterra form for Eq. \eqref{eq:macgeneral}. Assuming the nullclines $f_1(n_1,n_2)=0$ and $f_2(n_1,n_2)=0$ exist, exclusion is certain unless a type's nullcline extends out to greater densities in some region. In this example, the outcome depends on initial condition. (b) The constant-$N$, relative fitness description of selection.}
\end{figure}

Clearly MacArthur's argument is not compatible with the constant-$N$, relative fitness description of selection, in which $n_1$ and $n_2$ change along a line defined by $n_1+n_2=N$, and population density is otherwise irrelevant (Fig.~\ref{fig:Ksel}b). Since $N$ is fixed, it is not possible to persist at higher densities than $N$, nor is it possible to suppress the density of a competitor at densities lower than $N$ 

In this case, all four $K$ values are equal to $N$ [not!], and the $f_1$ and $f_2$ nullclines do not even exist.

Given the extremely broad variety of absolute population growth dynamics represented by Eq.~\eqref{eq:macgeneral}, at this point we might start to doubt that the relative fitness, constant=$N$ description of selection has a plausible basis in population ecology. In particular, the uncoupling of selection and population demography could be problematic, because they are both driven by the same birth/death events. 


In fact, one step in MacArthur's argument considerably narrows its scope: assuming that the nullclines of $f_1$ and $f_2$ exist at all. The strength of this assumption is clear if we note that essence of the dichotomy underlying the $r$/$K$ scheme is between interaction-dependent selection and interaction-independent selection. That is, selective shifts in frequency are a result of differences in absolute growth rates, but these differences can arise in two logically distinct ways: 1) some types expand more rapidly the absence of interactions between individuals or 2) some types are superior in their interactions with other types. Population density is a key factor controlling whether individuals interact, thereby setting the relative contributions of these forms of selection. But there are obviously important forms of interaction that are not expressible in terms of slanted density isoclines. For instance, contests for territory and mates are predominantly relative. MacArthur's argument starts with assumptions about how selection depends on density, and then tries to make sense of what this says about the evolution of competitive traits such as ``efficiency'' of consumable resource consumption. The study of density-dependent selection should start from a description of the interactions between types, and then analyze how the effects of those interactions vary with density. 

%This  also seems to be why the density-dependent selection literature expended such considerable effort to showing that evolution under density-dependent selection optimizes population size in some sense \citep{roughgarden_1979} --- a meaningless 


%In the case of exploitation competition for consumable resources, intra- and inter-type competition are connected via resource use ``efficiency'', but obviously this need not be true more generally \citep{gill_1974,case_1974}.


Here we investigate the 

territorial model of growth, dispersal and competition. 

We revisit the classic lottery model of \cite{chesson_1981}, which has two features that make it well suited for this role, but one critical flaw that we rectify here.

The first feature is that the lottery representation of competition is  particularly concise. Mature individuals (``adults'') each require their own territory, whereas newborn individuals (``propagules'') disperse to, and subsequently compete for, territories made available by the death of adults. Territorial contest among propagules leaves a single victorious adult per territory, the victor chosen at random from the propagules present, with probabilities weighted by a coefficient for each type representing competitive ability, akin to a lottery \citep{sale_77}.  

The second feature is the close connection between the lottery model and one of the foundational models of population genetics, the Wright-Fisher model of genetic drift, which we discuss further below. 

The critical flaw of the classic lottery model is that it breaks down at low densities (few propagules dispersing to each territory), precluding density-dependent behaviour. Our first task is to analytically extend the classic lottery model to correctly account for low density behavior (sections ``Model'' and ``Mean field approximation'').

Taking an example inspired by recent studies of rapid, seasonal evolution in \textit{Drosophila} \citep{bergland_14}, we discuss how environmental fluctuations might stabilize polymorphisms when population density is cyclical. 


 
\section*{Model}\label{sec:model}

We assume that reproductively mature individuals (``adults'') each require their own territory to survive and reproduce (Fig.~\ref{fig:lottery}). All territories are identical, and the total number of territories is $T$. Time $t$ advances in discrete iterations, each representing the time from birth to reproductive maturity. In iteration $t$, the number of adults of the $i$'th genotype is $n_i(t)$, the total number of adults is $N(t)=\sum_i n_i(t)$, and the number of unoccupied territories is $U(t)=T-N(t)$. 

We assume that the $n_i$ and $T$ are large enough that stochastic fluctuations in the $n_i$ (``drift'') can be ignored. We derive deterministic equations for the expected change in the $n_i$ over time, leaving the evaluation of drift for future work. This is an excellent approximation when the $n_i$ are all large. However, we also do not evaluate the initial stochastic behaviour of adaptive mutant lineages while they are at low abundance. When considering new mutations, we therefore restrict our attention to begin with the earliest (lowest $n_i$) deterministic behavior of mutant lineages (the transition to deterministic growth occurs at an abundance $n_i$ of order equal to their inverse expected absolute growth rate; \citealt{uecker_2011}).

\begin{figure}
\centering
\includegraphics[scale=0.8]{lottery.pdf}
\caption{\label{fig:lottery} Each iteration of our model has three elements. First, propagules are produced by adults and dispersed at random (only propagules landing on unoccupied territories are shown). Lottery competition then occurs in each unoccupied territory (only illustrated in one territory). Each genotype has a probability proportional to $c_i x_i$ of securing a given territory, where $c_i$ measures competitive ability and $x_i$ is the number of propagules that disperse there. In the illustrated territory, the black genotype disperses more propagules but is a poorer competitor. Territories are then made available by adult mortality (red crosses).}
\end{figure}

Each iteration, adults produce new offspring (``propagules''), $m_i$ of which disperse to unoccupied territories. We assume that adults cannot be ousted from their territories, so that $m_i$ only includes propagules landing on unoccupied territories. Propagules disperse at random over the unoccupied territories, regardless of distance from their parents, and independently of each other. There is no interaction between propagules (e.g. avoidance of territories crowded with propagules). Loss of propagules during dispersal is subsumed into $m_i$. 

In general, $m_i$ will increase with $n_i$, and will depend on population density $N$. For example, if $b_i$ is the number of successfully dispersing propagules produced per genotype $i$ adult, then the loss of propagules due to dispersal to occupied territories implies $m_i=b_i(1-N/T)n_i$, akin to Levins' competition-colonization model \citep{levins_71,tilman_94}. In section ``Cyclical birth and death rates'' we evaluate Eq. \eqref{eq:master} numerically using this functional form for $m_i$, with $b_i$ assumed to be constant. 

In ``Invasion of rare genotypes and coexistence'', we assume the simpler form $m_i=b_i n_i$, with constant $b_i$, meaning that all propagules land on unoccupied territories (a form of directed dispersal). This simplifies the mathematics without affecting the results of those sections, which only depend on the low-frequency invasion behavior of Eq. \eqref{eq:master}. Note that due to our assumption of uniform dispersal, the parameter $b_i$ can be thought of as a measure of ``colonization ability'', which combines fecundity and dispersal ability \citep{levins_71,tilman_94,bolker_99}. 

The number of individuals of the $i$'th genotype landing in any particular territory is denoted $x_i$. We assume that $x_i$ follows a Poisson distribution $p_i(x_i)=l_i^{x_i} e^{-l_i}/x_i!$, where $l_i=m_i/U$ is the mean territorial propagule density. This is approximation becomes exact when the $n_i$ are large enough that drift in $n_i$ can be ignored (Appendix A).

When multiple propagules land on the same territory, the victor is determined by lottery competition: genotype $i$ wins a territory with probability $c_i x_i/\sum_j c_j x_j$, where $c_i$ is a constant representing relative competitive ability (Fig. \ref{fig:lottery}). 

In the classic lottery model \citep{chesson_1981}, unoccupied territories are assumed to be saturated with propagules from every genotype $l_i\gg 1$. From the law of large numbers, the composition of propagules in each territory will then not deviate appreciably from the mean composition $l_1,l_2,\ldots,l_G$ ($G$ is the number of genotypes present), and so the probability that genotype $i$ wins any particular unoccupied territory is approximately $c_i l_i/\sum_j c_j l_j$. Let $\Delta_+ n_i$ denote the number of territories won by genotype $i$. Then $\Delta_+ n_1,\Delta_+ n_2,\ldots,\Delta_+ n_G$ follow a multinomial distribution with $U$ trials and success probabilities $\frac{c_1 l_1}{\sum_j c_j l_j},\frac{c_2 l_2}{\sum_j c_j l_j},\ldots,\frac{c_G l_G}{\sum_j c_j l_j}$, respectively. Genotype $i$ is expected to win $c_i l_i/\sum_j c_j l_j$ of the $U$ available territories, and deviations from this expected outcome are small (since $T$ is large by assumption), giving 
\begin{equation}
\Delta_+ n_i(t)=\frac{c_i l_i}{\sum_j c_j l_j}U(t)=b_i n_i\frac{1}{L}\frac{c_i}{\overline{c}}, \label{eq:lottery}
\end{equation}
where $\overline{c}=\sum_j c_j m_j/M$ is the mean propagule competitive ability for a randomly selected propagule, $L=M/U$ is the total propagule density and $M=\sum_j m_j$ is the total number of propagules. 

There is a close connection between the classic lottery model and the Wright-Fisher model of genetic drift \citep{svardal_2015}. In the Wright-Fisher model, genotype abundances are sampled each generation from a multinomial distribution with success probabilities $w_i n_i/\sum_j w_j n_j$, where $w$ is relative fitness and the $n_i$ are  genotype abundances in the preceding generation. Population size $N$ remains constant. This is mathematically equivalent to the classic lottery model with non-overlapping generations ($d_i=1$ for all $i$) and $w_i=b_i c_i$. Thus, the classic lottery model allows us to replace the abstract Wright-Fisher relative fitnesses $w_i$ with more ecologically-grounded fecundity, competitive ability and mortality parameters $b_i$, $c_i$ and $d_i$, respectively. Since birth and death rates affect absolute abundances, this allows us to evaluate selection at different densities (after appropriate extensions are made), in an otherwise very similar model to the canonical Wright-Fisher. We therefore expect that drift in realized values of $n_i$ in our extended lottery model should be similar to that in the Wright-Fisher model, but we leave this for future work. 

In our extension of the classic lottery model, we do not restrict ourselves to high propagule densities. Eq. \eqref{eq:lottery} is nonsensical if even a single type has low propagule density ($l_i\ll 1$): genotype $i$ can win at most $m_i$ territories, yet Eq. \eqref{eq:lottery} demands $c_i l_i/\sum_j c_j l_j$ of the $U$ unoccupied territories, for any value of $U$. Intuitively, the cause of this discrepancy is that individuals are discrete. Genotypes with few propagules depend on the outcome of contests in territories where they have at least one propagule present, not some small fraction of a propagule as would be implied by small $l_i$ in the classic lottery model. In other words, deviations from the mean propagule composition $l_1,l_2,\ldots,l_G$ are important at low density. 

We expect that a fraction $p_1(x_1)\ldots p_G(x_G)$ of the $U$ unoccupied territories will have the propagule composition $x_1,\ldots,x_G$. Genotype $i$ is expected to win $c_i x_i/\sum_j c_j x_j$ of these. Ignoring fluctuations about these two expectations (due to our no-drift, large $T$, large $n_i$ approximation), genotype $i$'s territorial acquisition is given by
\begin{equation}
\Delta_+ n_i(t)=U(t)\sum_{x_1,\ldots,x_G} \frac{c_i x_i}{\sum_j c_j x_j} p_1(x_1)\ldots p_G(x_G), \label{eq:growthsumuncoupled}
\end{equation}
in our extended lottery model, where the sum only includes territories with at least one propagule present. Note that unlike the classic lottery model, not all unoccupied territories are claimed each iteration, since under Poisson dispersal a fraction $e^{-L}$ remain unoccupied.

We assume that mortality only occurs in adults (Fig.~\ref{fig:lottery}; setting aside the juvenile deaths implicit in territorial contest), and at a constant, genotype-specific per-capita rate $0\leq d_i\leq 1$, so that the overall change in genotype abundances is
\begin{equation}
\Delta n_i(t)=\Delta_+ n_i(t)-d_i n_i(t). \label{eq:delttot}
\end{equation}

\section*{Results}

\subsection*{Mean Field Approximation}

Eq. \eqref{eq:growthsumuncoupled} involves an expectation over the time-dependent dispersal distributions $p_i$, and is thus too complicated to give intuition about the dynamics of density-dependent lottery competition. We now evaluate this expectation using a ``mean field'' approximation. 

Similarly to the high-$l_i$ approximation of classic lottery model, we replace the $x_i$ with appropriate mean values, although we cannot simply replace $x_i$ with $l_i$. For a genotype with low propagule density $l_i\ll 1$, we have $x_i=1$ in the territories where its propagules land, and so its growth comes entirely from territories which deviate appreciably from $l_i$. To account for this, we separate Eq. \eqref{eq:growthsumuncoupled} into $x_i=1$ and $x_i>1$ parts. Our more general mean field approximation only requires that there are no large discrepancies in competitive ability (i.e. we do not have $c_i/c_j\gg 1$ for any two genotypes). We obtain (details in Appendix B)
\begin{equation}
\Delta_+ n_i(t)\approx b_i n_i\left[e^{-L}+(R_i+A_i)\frac{c_i}{\overline{c}}\right], \label{eq:master}
\end{equation}
where
\begin{equation}
R_i=\frac{\overline{c}e^{-l_i}(1-e^{-(L-l_i)})}{c_i +\frac{\overline{c}L- c_il_i}{L-l_i}\frac{L-1+e^{-L}}{1-(1+L)e^{-L}}},\label{eq:Dr}
\end{equation}
and
\begin{equation}
A_i=\frac{\overline{c}(1-e^{-l_i})}{\frac{1-e^{-l_i}}{1-(1+l_i)e^{-l_i}}c_il_i+\frac{\overline{c}L- c_il_i}{L-l_i}\left(L\frac{1-e^{-L}}{1-(1+L)e^{-L}}-l_i\frac{1-e^{-l_i}}{1-(1+l_i)e^{-l_i}}\right)}.\label{eq:Da}
\end{equation}
To supplement our analytical mean field derivation, we did numerical simulations of our exact our density-dependent lottery model, and verified  that Eq. \eqref{eq:master} is a good approximation (Appendix B). Thus, Eq. \eqref{eq:master} describes how type abundances change over time in a lottery model where population density can itself vary with time.

Comparing Eq. \eqref{eq:master} to Eq. \eqref{eq:lottery}, the classic lottery per-propagule success rate $c_i/\overline{c}L$ has been replaced by three separate terms. The first, $e^{-L}$, accounts for propagules which land alone on unoccupied territories; these territories are won without contest. The second, $R_i c_i/\overline{c}$ represents competitive victories when the $i$ genotype is a rare invader in a high density population: from Eq. \eqref{eq:Dr}, $R_i\rightarrow 0$ when the $i$ genotype is abundant ($l_i\gg 1$), or other genotypes are collectively rare ($L-l_i\ll 1$). The third term, $A_ic_i/\overline{c}$, represents competitive victories when the $i$ genotype is abundant: $A_i\rightarrow 0$ if $l_i\ll 1$. The relative importance of these three terms varies with both the overall propagule density $L$ and the relative propagule frequencies $m_i/M$. If $l_i\gg 1$ for all genotypes, we recover the classic lottery model (only the $A_ic_i/\overline{c}$ term remains, and $A_i\rightarrow 1/L$). 

\subsection*{Coexistence in constant and cyclical environments}\label{sec:invas}

In the previous section we only considered how $b$, $c$ and $d$ should respond to selection in Grime's environmental extremes, based on invasion fitness. Here we further explore the low frequency behavior of Eq. \eqref{eq:master} to determine which types can coexist in a constant environment, and then consider the full time-dependent behaviour of Eq. \eqref{eq:master} in a cyclical environment. 

In a constant environment, stable coexistence is possible in our extended lottery model. A $b$-specialist $i$ and $c$-specialist $j$ ($b_i>b_j$, $c_j>c_i$) can co-exist because then propagule density $L$ is frequency-dependent, and so is the importance of competitive ability (Appendix D). This is a version of the classic competition-colonization trade-off \citep{tilman_94,levins_71}; the competitor ($c$-specialist) leaves many territories unoccupied (low $L$) due to its poor colonization ability (low $b$), which the colonizer ($b$-specialist) can then exploit. A similar situation holds for coexistence between high-$c$ and low-$d$ specialists; a ``competition-longevity'' trade-off \citep{tilman_94}. These forms of co-existence require density dependence (being mediated by $L$), and are not present in the classic lottery model. Coexistence is not possible between $b$- and $d$-specialists in a constant environment (Appendix D). 

Now suppose that birth and death rates vary periodically with amplitude sufficent to cause large changes in population density. This example is inspired by natural \textit{Drosophila} populations, which expand rapidly in the warmer months when fruit is abundant, but largely die off in the colder months. Along with this seasonal population density cycle, hundreds of polymorphisms exhibit frequency cycles that are in phase with the seasons \citep{bergland_14}. Some of these polymorphisms may be adaptive and potentially millions of years old, suggesting stable coexistence \citep{bergland_14,messer_2016}. Selection on allele frequencies thus occurs on the same time scale as population demography, a situation vastly more complicated than classical sweeps in demographically stable populations \citep{messer_2016}.

The classical population genetic treatment of fluctuating selection suggests that environmental fluctuations do not promote coexistence. Allele frequencies are successively multiplied by relative fitness values for each environmental iteration, and so two alleles favored in different environments can only stably coexist if the product of fitnesses for one type exactly equals the product for the other \citep{dempster_1955}. Thus, stable coexistence still requires frequency-dependent selection or heterozygote advantage (as is required in a constant environment). 

This classical argument overlooks two general mechanisms that promote coexistence in fluctuating environments \citep{messer_2016}. The first is the classic version of the storage effect, which occurs when part of the population is protected from selection (due to overlapping generations in the lottery model; \citealt{chesson_1981}). The second is the bounded population size effect of \cite{yi_2013}, which occurs when each environmental cycle involves growth from low to high density, with the time spent growing each cycle dependent on the fitness of the types present. 

Fig.~\ref{fig:fluctuatingselection}a-c shows the behavior of Eq. \eqref{eq:master} for an example where $b$ and $d$ cycle between zero and positive values (``summers'' with rapid growth and no mortality, and ``winters'' with mortality and no growth). Both the storage effect (adults are sheltered from selection during the summer growth phase) and the bounded density effect (expansion to high density occurs every cycle) are operating. Two types are present, a $b$-specialist, which is better at rapidly growing in the summer (higher $b$), and a $d$-specialist which is better at surviving the winter (lower $d$). Neither type has an advantage over a full environmental cycle, and they stably coexist. This is due to a combination of the storage and bounded density effects (recall that stable coexistence between $b$ and $d$ specialists was not possible in a constant environment). 

The classic lottery model (Eq. 1) fails to give co-existence for these parameters because expansion to carrying capacity occurs immediately at the start of the summer (Fig. \ref{fig:fluctuatingselection}d-f). As a result, coexistence requires that the winter survivor's $b$ must be about $5$ times smaller than required when we properly account for the growth in the abundance of each type using Eq. \eqref{eq:master} (keeping the other parameters the same; Fig.~\ref{fig:fluctuatingselection}g-i). Previous models of the promotion of genetic variation via the storage effect  \citep{ellner_1994} similarly assume that the total number of offspring per iteration is constant, and would produce a similar error. 

\begin{figure}
\centering
\includegraphics[scale=0.7]{fluctuatingselection.pdf}
\caption{\label{fig:fluctuatingselection} Stable coexistence between $b$ and $d$ specialists in a fluctuating environment requires a much greater $b$ advantage in the classic lottery model compared to our density-dependent extension of it when population density is seasonally cyclical. (a) Birth and death rates seasonally alternate being nonzero (white for winter, green for summer). The $b$-specialist (black) has higher $b$ and $d$ ($b=0.5$, $d=0.2$) than the $d$-specialist ($b=0.217$, $d=0.1$) (blue). (b) Both types grow during the positive $b$ phase, and decline during the positive $d$ phase, but the $d$-specialist does so at a lower rate. Total height (blue+black) is population density $N/T$. (c) Summer favors the $b$ specialist, winter the $d$-specialist, and they stably coexist. (d-f) Same as (a-c) for the classic lottery model; the types no longer coexist. (g-i) Same as (d-f) where now $b = 0.0421$ for the $d$ specialist and the types coexist. For illustration, the propagule abundances are assumed to have the form $m_i=b_i(1-N/T)n_i$, reflecting non-directed dispersal.} 
\end{figure}

\section*{Discussion}

It is interesting to compare the predictions of the extended lottery model with earlier approaches, such as the $r$/$K$ scheme, where $r=b-d$ is the maximal, low-density growth rate \citep{pianka_1972}. Confusingly, the term ``$K$-selection'' sometimes refers generally to selection at high density \citep{pianka_1972}, encompassing both selection for higher saturation density \citep{macarthur_1967} and competitive ability \citep{gill_1974}. Contrary to predictions of an $r$/$K$ trade-off, empirical studies have shown that maximal growth rate at low density and the high density at which saturation occurs (measured by abundance) are positively correlated, both between species/strains \citep{luckinbill_1979,kuno_1991,hendriks_2005,fitzsimmons_2010}, and as a result of experimental evolution \citep{luckinbill_1978,luckinbill_1979}. From the perspective of our model, this positive correlation is not surprising since the saturation density, which is determined by a balance between births and deaths, increases with $b$. 

%This prediction is also counter to the expectations of MacArthur's $r$/$K$ dichotomy \citep{macarthur_1967}, since $b$ is closely related to the maximal, low-density growth rate $r=b-d$ \citep{pianka_1972}, yet in the $r$/$K$ scheme, high density populations should be subject to $K$, not $r$, selection. Yet it is not surprising that $b$ can matter at high densities. In our model (or any lottery model of competition), $b$ matters at high densities because territorial contests among juveniles are intrinsically unpredictable. This is a realistic feature of the model. Even if one genotype is guaranteed to win a territory in a ``fair'' contest (e.g. it is the most efficient exploiter of a limiting consumable resource; \citealt{tilman_1982}), inferior competitors can win by chance. For example, an inferior competitor's propagules may happen to arrive first, gaining a decisive developmental advantage. First arrivals are more likely to occur for genotypes with a fecundity and/or dispersal advantage, as represented by higher $b$ in lottery models. The analogous intuition in the Wright-Fisher model is that fecundity confers a relative fitness advantage, even though population size is not changing. The logistic model for which $r$ and $K$ are named, does not capture this intuition. 

There is support for a negative relationship between competitive success at high density and maximal growth rate \citep{luckinbill_1979}, consistent with a tradeoff between $r$ and the competitive aspect of $K$. This could be driven by a tradeoff between individual size and reproductive rate. To avoid confusion with other forms of ''$K$-selection'', selection for competitive ability has been called ``$\alpha$-selection'' after the competition coefficients in the Lotka-Volterra equation \citep{gill_1974,case_1974,joshi_2001}. However, competitive success as measured by $\alpha$ (i.e. the per-capita effect of one genotype on another genotype's growth rate) is only partly determined by individual competitive ability --- in the presence of age-structured competition and territoriality, it also includes the ability of each genotype to produce contestants i.e. $b$ in our model. Our $c$ is strictly competitive ability only --- as such, changes in $c$ do not directly affect population density (the total number of territories occupied per iteration is $\Delta_+ N=U(1-e^{-L})$, which does not depend directly on the $c_i$). The clean separation of a strictly-relative $c$ parameter is particularly useful from an evolutionary genetics perspective, essentially embedding a zero-sum relative fitness trait within a non-zero-sum fitness model. This could have interesting applications for modeling the impacts of intra-specific competition on species extinction, for example due to clonal interference \citep{gerrish_1998,desai_2007} between $c$-strategists on the one hand, and $b$- and $d$- strategists on the other.

$K$-selection in the narrow logistic sense of selection for a greater environmental carrying capacity for given $r$, sometimes referred to as ``efficiency'' \citep{macarthur_1967}, could be represented in our model by smaller individual territorial requirements. To a first approximation, two co-occurring genotypes which differ by a small amount in their territorial requirements only should have the same fitness, since the costs or benefits of a change in the amount of unocupied territory is shared equally among genotypes via the propagule density per territory $L$. The situation is more complicated when the differences in territorial requirements become large enough that territorial contests can occur on different scales for different genotypes. We leave these complications for future work. 

Nevertheless, it is interesting to note that ruderals, which are typically thought of as high fecundity dispersers ($b$-specialists), may also be strongly $d$-selected, which while unintuitive, is consistent with our findings. An effective way to reduce $d$ in the face of unavoidable physical destruction is to shorten the time to reproductive maturity --- short life cycles are a characteristically ruderal trait. Moreover, a recent hierarchical cluster analysis of coral traits did find a distinct ``ruderal'' cluster, but high fecundity was not its distinguishing feature. Rather, ruderals used brood- (as opposed to broadcast-) spawning, which could plausibly be a mechanism for improving propagule survivorship in disturbed environments \citep{darling_2012}. 

One potential limitation of our model as a general-purpose model of density-dependent selection is its restriction to interference competition between juveniles for durable resources (lottery recruitment to adulthood), analogous to the ubiquitous assumption of viability selection in population genetics \citep[p. 45]{ewens_2004}. In some respects this is the complement of consumable resource competition models, which restrict their attention to indirect exploitation competition, typically without age structure \citep{tilman_1982}. In the particular case that consumable resources are spatially localized (e.g. due to restricted movement through soils), resource competition and territorial acquisition effectively coincide, and in principle resource competition could be represented by a competitive ability $c$ (or conversely, $c$ should be derivable from resource competition). The situation is more complicated if the resources are well-mixed, since, in general, resource levels then need to be explicitly tracked. It seems plausible that explicit resource tracking may not be necessary when the focus is on the evolution of similar genotypes that use identical resources rather than the stable co-existence of widely differing species with different resource preferences \citep{ram_2016}. We are not aware of any attempts to delineate conditions under which explicit resource tracking is unnecessary even if it is assumed that community structure is ultimately determined by competition for consumable resources. More work is needed connecting resource competition models to the density-dependent selection literature, since most of the former has to date been focused on narrower issues of the role of competition at low resource availability and in the absence of direct interactions between organisms at the same trophic level \citep{aerts_1999,davis_1998,tilman_2007}.  

While our model can be applied to species rather than genotypes (e.g. ecological invasions), our focus is genotype evolution i.e. the change in allele frequencies over time. Our assumption that there are no large $c$ discrepancies (section ``Mean field approximation'') amounts to a restriction on the amount of genetic variation in $c$ in the population. Since beneficial mutation effect sizes will typically not be much larger than a few percent, large $c$ discrepancies can only arise if the mutation rate is extremely large, and so the assumption will not be violated in most cases. However, this restriction could become important when looking at species interactions rather than genotype evolution.

In the introduction we mentioned the recurring difficulties with confounding selection and demography in population genetic inference. It seems that Eq. \eqref{eq:master} or something similar (and hopefully more analytically tractable) is unavoidable for the analysis of time-course genetic data because, fundamentally, selective births and deaths affect both abundances and frequencies, not one or the other in isolation. Moreover, some aspects of allele frequency change are intrinsically density-dependent. In the classic lottery model, which as we have seen is essentially the Wright-Fisher model with overlapping generations, $b_i$ and $c_i$ are  equivalent in the sense that the number of territorial victories only depends on the product $b_i c_i$ (see ``Model''). This is no longer the case in our extension, where $b$ and $c$ specialists can co-exist. This ``colonization-competition trade-off'' is well known in the co-existence literature \citep{tilman_94}. It and similar forms of ``spatial co-existence'' in stable environments have previously been modeled either with Levin's  qualitative representation of competition \citep{levins_71,tilman_94}, as opposed to the quantitative $c$ of lottery competition, or with a more sophisticated treatment of space (non-uniform dispersal; \citealt{shmida_84,bolker_99}). In cyclical environments, polymorphisms can be stabilized by the bounded density effect, which is  completely lost if there is an exclusive focus on allele frequencies \citep{yi_2013}. We leave the details of how our model might be applied to inference problems, including the crucial issue of its genetic drift predictions (providing a null model for neutral sites), for future work. 

%\section*{Acknowledgments}

%We thank Peter Chesson and Joachim Hermisson for many constructive comments on this manuscript. This work was financially supported by the National Science Foundation (DEB-1348262).

\bibliographystyle{plainnat}
\bibliography{reference} 

\section*{Appendix A: Poisson approximation}

For simplicity of presentation, we have assumed a Poisson distribution for the $x_i$ as our model of dispersal. Strictly speaking, the total number of $i$ propagules $\sum x_i$ (summed over unoccupied territories) is then no longer a constant $m_i$, but fluctuates between generations for a given mean $m_i$, which is more biologically realistic. Nevertheless, since we do not consider the random fluctuations in type abundances here, and for ease of comparison with the classic lottery model, we ignore the fluctuations in $m_i$. Instead we focus, on Poisson fluctuations in propagule composition in each territory. 

In the exact model of random dispersal, the counts of a genotype's propagules across unnocupied territories follows a multinomial distribution with dimension $U$, total number of trials equal to $m_i$, and equal probabilities $1/U$ for a propagule to land in a given territory. Thus, the $x_i$ in different territories are not independent random variables. However, for sufficiently large $U$ and $m_i$, this multinomial distribution for the $x_i$ across territories is closely approximated by a product of independent Poisson distributions for each territory, each with rate parameter $l_i$ \citep[Theorem 1]{arenbaev_1977}. Since we are ignoring finite population size effects, we effectively have $T\rightarrow \infty$, in which case $U$ can be only be small enough to violate the Poisson approximation if there is vanishing population turnover, and then the dispersal distribution is irrelevant anyway. Likewise, in ignoring stochastic finite population size for the $n_i$, we have effectively already assumed that $m_i$ is large enough to justify the Poisson approximation (the error scales as $1/\sqrt{m_i}$; \citealt{arenbaev_1977}).

\section*{Appendix B: Derivation of growth equation}

We separate the right hand side of Eq.~\eqref{eq:growthsumuncoupled} into three components $\Delta_+ n_i = \Delta_u n_i+\Delta_r n_i+\Delta_a n_i$ which vary in relative magnitude depending on the propagule densities $l_i$. Following the notation in the main text, the Poisson distributions for the $x_i$ (or some subset of the $x_i$) will be denoted $p$, and we use $P$ as a general shorthand for the probability of particular outcomes.

\subsection*{Growth without competition}

The first component, $\Delta_u n_i$, accounts for territories where only one focal propagule is present $x_i=1$ and $x_j=0$ for $j\neq i$ ($u$ stands for ``uncontested''). The proportion of territories where this occurs is $l_i e^{-L}$, and so 
\begin{equation}
\Delta_u n_i=Ul_i e^{-L}=m_i e^{-L}.
\end{equation}

\subsection*{Competition when rare}

The second component, $\Delta_r n_i$, accounts for territories where a single focal propagule is present along with at least one non-focal propagule ($r$ stands for ``rare'') i.e. $x_i=1$ and $X_i\geq 1$ where $X_i=\sum_{j\neq i} x_j$ is the number of nonfocal propagules. The number of territories where this occurs is $Up_i(1)P(X_i\geq 1)=b_i n_i e^{-l_i}(1-e^{-(L-l_i)})$. Thus 
\begin{equation}
\Delta_r n_i = m_i e^{-l_i}(1-e^{-(L-l_i)})\left\langle  \frac{c_i}{c_i +\sum_{j\neq i} c_j x_j } \right\rangle_{\tilde{p}},  \label{eq:deltr}
\end{equation}
where $\langle \rangle_{\tilde{p}}$ denotes the expectation with respect to $\tilde{p}$, and $\tilde{p}$ is the probability distribution of nonfocal propagule abundances $x_j$ \textit{after} dispersal, in those territories where exactly one focal propagule, and at least one non-focal propagule, landed. 

Our ``mean field'' approximation is to replace $x_j$ with its mean in the last term in Eq.~\eqref{eq:deltr},
\begin{equation}
\left\langle\frac{c_i}{c_i +\sum_{j\neq i} c_j x_j}\right\rangle_{\tilde{p}}\approx \frac{c_i}{c_i +\sum_{j\neq i} c_j \langle x_j\rangle_{\tilde{p}}}.\label{eq:meanfieldr}
\end{equation}
Below we justify this replacement by arguing that the standard deviation $\sigma_{\tilde{p}}(\sum_{j\neq i} c_j x_j)$ (with respect to $\tilde{p}$), is much smaller than $\langle\sum_{j\neq i} c_j x_j\rangle_{\tilde{p}}$.

We first calculate $\langle x_j \rangle_{\tilde{p}}$. Let $X=\sum_j x_j$ denote the total number of propagules in a territory and ${\mathbf x_i}=(x_1,\ldots,x_{i-1},x_{i+1}\ldots,x_G)$ denote the vector of non-focal abundances, so that $p({\mathbf x_i})=p_1(x_1)\ldots p_{i-1}(x_{i-1})p_{i+1}(x_{i+1})\ldots p_G(x_G)$. Then, $\tilde{p}$ can be written as
\begin{align}
\tilde{p}({\mathbf x_i})&=p({\mathbf x_i}|X\geq 2,x_i=1)\nonumber\\
&=\frac{P({\mathbf x_i},X\geq 2|x_i=1)}{P(X\geq 2)}\nonumber\\
&=\frac{1}{1-(1+L)e^{-L}}\sum_{X=2}^{\infty} P(X) p({\mathbf x_i}|X_i=X-1),
\end{align}
and so
\begin{align}
\langle x_j \rangle_{\tilde{p}}&=\sum_{\mathbf x_i} \tilde{p}({\mathbf x_i})x_j\nonumber\\
&=\frac{1}{1-(1+L)e^{-L}}\sum_{X=2}^{\infty} P(X) \sum_{\mathbf x_i} p({\mathbf x_i}|X_i=X-1)x_j.
\label{eq:raremonster1}
\end{align}
The inner sum over ${\mathbf x_i}$ is the mean number of propagules of a given nonfocal type $j$ that will be found in a territory which received $X-1$ nonfocal propagules in total, which is equal to $\frac{l_j}{L-l_i}(X-1)$. Thus, 
\begin{align}
\langle x_j \rangle_{\tilde{p}}&=\frac{l_j}{1-(1+L)e^{-L}}\frac{1}{L-l_i}\sum_{k=2}^{\infty} P(X) (X-1)\nonumber\\
&=\frac{l_j}{1-(1+L)e^{-L}}\frac{L-1+e^{-L}}{L-l_i},
\label{eq:meanxjrare}
\end{align}
where the last line follows from $\sum_{X=2}^{\infty} P(X)(X-1)=\sum_{X=1}^{\infty} P(X)(X-1)=\sum_{X=1}^{\infty} P(X)X-\sum_{X=1}^{\infty}P(X)$.

The exact analysis of the fluctuations in $\sum_{j\neq i} c_j x_j$ is complicated because the $x_j$ are not independent with respect to $\tilde{p}$. These fluctuations are part of the ``drift'' in type abundances which we leave for future work. Here we use the following approximation to give some insight into the magnitude of these fluctuations and also the nature of the correlations between the $x_j$. We replace $\tilde{p}$ with $\tilde{q}$, defined as the ${\mathbf x_i}$ Poisson dispersal probabilities conditional on $X_i\geq1$ (which are independent). The distinction between $\tilde{p}$ with $\tilde{q}$ will be discussed further below. The $\tilde{q}$ approximation gives $\langle x_j \rangle_{\tilde{q}}=\langle x_j \rangle_p/C=l_j/C$, 
\begin{align}
\sigma_{\tilde{q}}^2(x_j)&=\langle x_j^2 \rangle_{\tilde{q}}-\langle x_j \rangle_{\tilde{q}}^2\nonumber\\
&=\frac{1}{C}\langle x_j^2 \rangle_p-\frac{l_j^2}{C^2}\nonumber \\
&=\frac{1}{C}(l_j^2 + l_j)-\frac{l_j^2}{C^2}\nonumber \\
&=\frac{l_j^2}{C}\left(1-\frac{1}{C}\right)+\frac{l_j}{C},\label{eq:varr}
\end{align}
and 
\begin{align}
\sigma_{\tilde{q}}(x_j,x_k)&=\langle x_j x_k \rangle_{\tilde{q}}-\langle x_j \rangle_{\tilde{q}}\langle x_k \rangle_{\tilde{q}}\nonumber\\
&=\frac{1}{C}\langle x_j x_k \rangle_p-\frac{l_jl_k}{C^2}\nonumber\\
&=\frac{l_j l_k}{C}\left(1-\frac{1}{C}\right),\label{eq:covr}
\end{align}
where $C=1-e^{-(L-l_i)}$ and $j\neq k$. 

The exact distribution $\tilde{p}$ assumes that exactly one of the  propagules present in a given site after dispersal belongs to the focal type, whereas $\tilde{q}$ assumes that there is a focal propagule present before non-focal dispersal commences. As a result, $\tilde{q}$ predicts that the mean propagule density is greater than $L$ (in sites with only one focal propagule is present) when the focal type is rare and the propagule density is high. This is erroneous, because the mean number of propagules in every site is $L$ by definition. Specifically, if $L-l_i \approx L\gg 1$, then the mean propagule density predicted by $\tilde{q}$ is approximately $L+1$. The discrepancy causes rare invaders to have an intrinsic rarity disadvantage (territorial contests under $\tilde{q}$ are more intense than they should be). In contrast, Eq. \eqref{eq:meanxjrare} correctly predicts that there are on average $\sum_{j\neq i}\langle x_j \rangle_{\tilde{p}}\approx L-1$ nonfocal propagules because $\tilde{p}$ accounts for potentially large negative covariances between the $x_j$ ``after dispersal''. By neglecting the latter covariences, $\tilde{q}$ overestimates the fluctuations in $\sum_{j\neq i} c_j x_j$; thus $\tilde{q}$ gives an upper bound on the fluctuations. The discrepancy between $\tilde{q}$ and $\tilde{p}$ will be largest when $L$ is of order $1$ or smaller, because then the propagule assumed to already be present under $\tilde{q}$ is comparable to, or greater than, the entire propgaule density. 

Decomposing the variance in $\sum_{j\neq i} c_j x_j$,
\begin{equation}
\sigma_{\tilde{q}}^2(\sum_{j\neq i} c_j x_j)=\sum_{j\neq i}\left[c_j^2\sigma_{\tilde{q}}^2(x_j)+2\sum_{k>j, k\neq i}c_j c_k\sigma_{\tilde{q}}(x_j,x_k)\right],\label{eq:vartotr}
\end{equation}
and using the fact that $\sigma_{\tilde{q}}(x_j,x_k)$ and the first term in Eq. \eqref{eq:varr} are negative because $C<1$, we obtain an upper bound on the relative fluctuations in $\sum_{j\neq i} c_j x_j$, 
\begin{equation}
\frac{\sigma(\sum_{j\neq i} c_j x_j)}{\langle\sum_{j\neq i} c_j x_j\rangle}=C^{1/2}\frac{\left(\sum_{j\neq i}c_j^2 l_j+(1-1/C)\left(\sum_{j\neq i}c_j l_j\right)^2 \right)^{1/2}}{\sum_{j\neq i}c_j l_j}<C^{1/2}\frac{\left(\sum_{j\neq i}c_j^2 l_j\right)^{1/2}}{\sum_{j\neq i}c_j l_j}. \label{eq:cvr}
\end{equation}

Suppose that the $c_j$ are all of similar magnitude (their ratios are of order one). Then Eq.~\eqref{eq:cvr} is $\ll 1$ for the case when $L-l_i \ll 1$ (due to the factor of $C^{1/2}$), and also for the case when at least some of the nonfocal propagule densities are large $l_j\gg 1$ (since it is then of order $1/\sqrt{L-l_i}$). The worst case scenario occurs when $L-l_i$ is of order one. Then Eq.~\eqref{eq:cvr} gives a relative error of approximately $50\%$, which from our earlier discussion we know to be a substantial overestimate when $L$ is of order $1$. Our numerical results (Fig. \ref{fig:simcomp}) confirm that the relative errors are indeed small.

However, the relative fluctuations in $\sum_{j\neq i} c_j x_j$ can be large if some of the $c_j$ are much larger than the others. Specifically, in the presence of a rare, extremely strong competitor ($c_j l_j\gg c_{j'} l_{j'}$ for all other nonfocal genotypes $j'$, and $l_j\ll 1$), then the RHS of Eq. \eqref{eq:cvr} can be large and we cannot make the replacement Eq.~\eqref{eq:meanfieldr}. 

Substituting Eqs. \eqref{eq:meanfieldr} and \eqref{eq:meanxjrare} into Eq.~\eqref{eq:deltr}, we obtain
\begin{equation}
\Delta_r n_i\approx m_i R_i\frac{c_i}{\overline{c}}, \label{eq:deltrfinal}
\end{equation}
where $R_i$ is defined in Eq.~\eqref{eq:Dr}.

\subsection*{Competition when abundant}

The final contribution, $\Delta_a n_i$, accounts for territories where two or more focal propagules are present ($a$ stands for ``abundant"). Similarly to Eq.~\eqref{eq:deltr}, we have 
\begin{equation}
\Delta_a n_i=U(1-(1+l_i)e^{l_i})\left\langle \frac{c_i x_i}{\sum_j c_j x_j} \right\rangle_{\hat{p}}\label{eq:delta}
\end{equation}
where $\hat{p}$ is the probability distribution of both focal and nonfocal propagaule abundances \textit{after} dispersal in those territories where at least two focal propagules landed. 

Again, we argue that the relative fluctuations in $\sum c_j x_j$ are much smaller than $1$ (with respect to $\hat{p}$), so that,
\begin{equation}
\left\langle \frac{c_i x_i}{\sum_j c_j x_j} \right\rangle_{\hat{p}}\approx  \frac{c_i \langle x_i \rangle_{\hat{p}}}{\sum_j c_j \langle x_j\rangle_{\hat{p}}}.\label{eq:meanfielda}
\end{equation}
Following a similar procedure as for $\Delta_r n_i$, where the vector of propagule abundances is denoted ${\mathbf x}$, the mean focal genotype abundance is, 
\begin{align}
\langle x_i \rangle_{\hat{p}}&=\sum_{\mathbf x} x_i p(\mathbf x|x_i\geq 2)\nonumber \\
&=\sum_{x_i} x_i p(x_i|x_i\geq 2) \nonumber\\
&=\frac{1}{1-(1+l_i)e^{-l_i}}\sum_{x_i\geq 2} p(x_i)x_i\nonumber\\
&=l_i\frac{1-e^{-l_i}}{1-(1+l_i)e^{-l_i}}.
\end{align}
For nonfocal genotypes $j\neq i$, we have
\begin{align}
\langle x_j \rangle_{\hat{p}}&=\sum_{\mathbf x} x_j p(\mathbf x|x_i\geq 2)\nonumber \\
&=\sum_{X}P(X|x_i\geq 2)\sum_{\mathbf x} x_j p({\mathbf x}|x_i\geq 2,X)\nonumber\\
&=\sum_{X}P(X|x_i\geq 2)\sum_{x_i} p(x_i|x_i\geq 2,X) \sum_{\mathbf x_i} x_j p(\mathbf x_i|X_i=X-x_i)\nonumber\\
&=\sum_{X}P(X|x_i\geq 2)\sum_{x_i}p(x_i|x_i\geq 2,X) \frac{l_j(X-x_i)}{L-l_i} \nonumber\\
&=\frac{l_j}{L-l_i}\left[\sum_{X}P(X|x_i\geq 2)X - \sum_{x_i}p(x_i|x_i\geq 2) x_i \right]\nonumber\\
&=\frac{l_j}{L-l_i}\left( L\frac{1-e^{-L}}{1-(1+L)e^{-L}}- l_i\frac{1-e^{-l_i}}{1-(1+l_i)e^{-l_i}}\right). 
\end{align}

To calculate the relative fluctuations in $\sum_{j\neq i} c_j x_j$, we use a similar approximation as for $\Delta_r n_i$: $\hat{p}$ is approximated by $\hat{q}$, defined as the ${\mathbf x}$ dispersal probabilities in a territory conditional on $x_i>2$ (that is, treating the $x_j$ as indepenent). All covariances between nonfocal genotypes are now zero, so that $\sigma_{\hat{q}}^2(\sum c_j x_j)=\sum c_j^2 \sigma_{\hat{q}}^2(x_j)$, where $\sigma_{\hat{q}}^2(x_j)=l_j$ for $j\neq i$, and  
\begin{equation}
\sigma_{\hat{q}}^2(x_i)=\frac{l_i}{D}\left(l_i+1-e^{-l_i}-\frac{l_i}{D}\left(1-e^{-l_i}\right)^2\right),
\end{equation}
where $D= 1-(1+l_i)e^{-l_i}$, and 
\begin{equation}
\frac{\sigma_{\hat{q}}(\sum c_j x_j)}{\langle\sum c_j x_j\rangle} = \frac{\left(\sum_{j\neq i} c_j^2 l_j + c_i^2 \sigma_{\hat{q}}^2(x_i)\right)^{1/2}}{\sum_{j\neq i} c_j l_j + c_i l_i (1-e^{-l_i})/D} \label{eq:cva}.
\end{equation}

Similarly to Eq.~\eqref{eq:cvr}, the RHS of Eq. \eqref{eq:cva} is $\ll 1$ for the case that $L \ll 1$ (due to a factor of $D^{1/2}$), and also for the case when at least some of the propagule densities (focal or nonfocal) are large --- provided that $c_i$ and the $c_j$ are all of similar magnitude. Again, the worst case scenario occurs when $l_i$ and $L-l_i$ are of order $1$, in which case Eq. \eqref{eq:cva} is around $35\%$, which is again where the $\hat{q}$ approximation produces the biggest overestimate of the fluctuations in ${\mathbf x}$. Similarly to Eq.~\eqref{eq:cvr}, the RHS of \eqref{eq:cva} will not be $\ll 1$ in the presence of a rare, extremely strong competitor.  

Combining Eqs. \eqref{eq:delta} and \eqref{eq:meanfielda}, we obtain
\begin{equation}
\Delta_a n_i=m_i A_i \frac{c_i}{\overline{c}},
\end{equation}
where $A_i$ is defined in Eq.~\eqref{eq:Da}.

\subsection*{Comparison with simulations}

Fig.~\ref{fig:simcomp} shows that Eq. \eqref{eq:master} and its components closely approximate our density-dependent lottery model over a wide range of propagule densities (the latter is evaluated by direct simulations of uniform random dispersal and lottery competition). Two genotypes are present, one of which is at low frequency. The growth of the low-frequency genotype relies crucially on the low-density competition term $R_i c_i/\overline{c}$, and also to a lesser extent on the high density competition term $A_i c_i/\overline{c}$ if $l_1$ is large enough (Fig.~\ref{fig:simcomp}b). On the other hand, $R_i c_i/\overline{c}$ is negligible for the high-frequency genotype, which depends instead on high density territorial victories (Fig.~\ref{fig:simcomp}d). Fig. 3 also shows the breakdown of the classic lottery model at low propagule densities.

\begin{figure}
\centering
\includegraphics[scale=0.7]{simulationcomparison.pdf}
\caption{\label{fig:simcomp} The change in genotype abundances in a density dependent lottery model is closely approximated by Eq. \eqref{eq:master}. $\Delta_+ n_i/m_i$ from Eq. \eqref{eq:master} (and its separate components) are shown, along with direct simulations of random dispersal and lottery competition over one iteration over a range of propagule densities ($U$ is varied between $5\times 10^3$ and $10^6$ with $m_1=10^4$ and $m_2=9\times 10^4$). Two genotypes are present. (a) and (b) show the low-frequency genotype with $c$-advantage ($c_1=1.5$), (c) and (d) show the high-frequency predominant genotype ($c_2=1$). Simulation points are almost invisible in (c) and (d) due to near exact agreement with Eq. \eqref{eq:master}. Dashed lines in (a) and (c) show the breakdown of the classic lottery model.} 
\end{figure}

\section*{Appendix C: Mutant invasion and coexistence in a constant environment}

Here we evaluate the initial growth or decline of mutants in a population with a single resident type, which is in equilibrium. To determine whether coexistence is possible, we check for ``mutual invasion'', that is, we check that type $j$ will invade an $i$-dominated population, but type $i$ will also invade a $j$-dominated population. 

Solving for equilibrium when $i$ is the resident ($\Delta n_i = 0$), we have $R_i=0$, $\overline{c}=c_i$, $A_i=(1-(1+L)e^{-L})/L$, and Eq. \eqref{eq:master} becomes
\begin{equation}
b_i(1-e^{-L})/L-d_i=0.\label{eq:equil}
\end{equation}
This implies $L\approx b_i/d_i$ if $b_i/d_i\gg 1$ and $L\ll 1$ if $b_i/d_i\approx 1$. 

Now suppose that a novel mutant $j$, which is initially rare, appears in the population. Then $A_j/R_j\ll 0$, $l_j\approx 0$ and $\overline{c}\approx c_i$, and so, from Eq. \eqref{eq:master}, the mutant lineage's fitness is
\begin{equation}
\Delta n_j/n_j \approx b_j \left(e^{-L}+R_j\frac{c_j}{c_i}\right)-d_j \label{eq:invad}
\end{equation}
where $R_j\approx (1-e^{-L})/\left(\frac{c_j}{c_i}+\frac{L-1+e^{-L}}{1-(1+L)e^{-L}}\right)$.

We consider the case of coexistence between a $b$-specialist $i$ and a $c$-specialist $j$ ($b_i>b_j$, $c_j>c_i$ and $d_i=d_j$). Suppose that $b_i$ is so large that $L\gg 1$ when $i$ is dominant, and $b_j$ is so small that $L\ll 1$ when $j$ is dominant. Then, when $j$ is dominant, we have $\Delta n_i/n_i=b_i-d_i=b_i-d_j=b_i-b_j>0$. When $i$ is dominant, Eq.  \eqref{eq:highLinvad} applies, where Eq. \eqref{eq:equil} implies $d_j=d_i=b_i(1-e^{-L})/L\approx b_i/L$, and so
\begin{equation}
\Delta n_j/n_j \approx \frac{b_j}{L}\frac{c_j}{c_i}-\frac{b_i}{L}.
\end{equation}
Therefore, coexistence occurs if $c_j/c_i$ is sufficiently large. The analogous argument for $d$- and $c$-specialists ($d_i<d_j$ with $L\gg 1$ when $i$ dominates, $L\ll 1$ when $j$ dominates, and $b_i=b_j$) gives $\Delta n_j/n_j \approx d_i\frac{c_j}{c_i}-d_j$, which again implies coexistence if $c_j/c_i$ is sufficiently large.

For $b$-and $d$-specialists ($c_i=c_j$), we have $\Delta n_j/n_j\approx b_j d_i/b_i-d_j$ when $i$ dominates and $\Delta n_i/n_i\approx b_i d_j/b_j-d_i$ when $j$ dominates. Thus, either $i$ or $j$ grows when rare, but not both, and stable coexistence is not possible in a constant environment.


\end{document}
